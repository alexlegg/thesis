\chapter{Introduction}

We rely on software systems to perform important tasks for us on a daily basis. Unfortunately we also frequently experience the frustration of an incorrect software system. However, as these systems become more ingrained into our lives the cost of incorrectness can be far greater than mere frustration.

In 1996 the European Space Agency lost their Ariane 5 rocket forty seconds after launch to an incorrect conversion from floating point to integer~\cite{Dowson97}. The cost of the failure was \$370 million in USD. More recently, Toyota has been forced to recall a large number of vehicles due to a failure in the software controlling the brakes~\cite{Parrish13}. The failures led to loss of life~\cite{CBS10}.

As the desire for software and the consequences of incorrectness has grown, the need for a systematic methodology for producing correct software has become apparent. One solution has been to develop strict engineering practices, including rigorous testing, to reduce the chance of errors. Another solution is to produce a proof of correctness of the software, either with or without the aid of a mechanised proof assistant. Model checking can also be used in some cases to fully automate the correctness proof.

A step further is to have our software automatically constructed for us, a technique first formally considered by Alonzo Church in the middle of the last century~\cite{Church62}. Software synthesis shifts the role of the developer from writing code to writing formal specifications. This completely eradicates the human error factor from the low level construction of software and allows developers to focus on high level system design. In all other approaches to software correctness the software must first be constructed; a process involving considerable time and effort.

Unfortunately, automatic software synthesis involves nontrivial computation. In broad strokes, the synthesis algorithm must determine how the state of the system is affected by the software and its environment and then select actions for the software such that no matter the actions of the environment the system adheres to the specification. In practice, on certain system specifications the process can lead to significant \emph{state explosion} that renders synthesis infeasible.

The state of the art in synthesis contains several methodologies that act as countermeasures to state explosion. However, no single approach is suited to all classes of specifications. In this thesis I propose an addition to the state of the art that is well suited to specifications with certain properties.

\section{Synthesis}

The usual formalisation of the synthesis problem is as a game between the software and its environment. This thesis is concerned with synthesis of controllers for safety games in which the controller must enforce that the game remains within a set of safe states. The usual approach to solving safety games is to iteratively construct a set of winning states that are known to be safe regardless of the actions of the environment. Once that set exists a winning strategy can be formed by choosing actions that have successor states within the winning region.

Explicit enumeration of the states in the winning region is infeasible even on small specifications so the set of states is usually represented symbolically. This is done by specifying the game with states as valuations of a set of boolean variables and using boolean algebra to symbolically define sets of states. Traditionally binary decision diagrams (BDDs) are used to represent boolean functions because they provide compact representations in most cases and there are efficient algorithms for operating on formulas in BDD form. The disadvantage of this approach is that in the worst case the representation occupies space that is exponential in the number of variables in the formula. A BDD is a canonical representation of a formula so it may be the case that a compact BDD representing the winning region for a particular specification does not exist.

Other approaches rely on satisfiability solvers to efficiently perform the operations required by synthesis on sets of states. The satisfiability problem (SAT) is the question of whether a value can be assigned to all variables in a formula such that the formula evaluates to true. Modern SAT solvers provide efficient implementations of backtracking search with computational learning that operate on boolean formulas in clause normal form (CNF). The advantage of a SAT based approach is that CNF is not canonical so in cases when a BDD cannot compactly represent a set of states it may be possible to do so in CNF.

The disadvantage of SAT based approaches is that solvers only determine whether a satisfying assignment to variables \emph{exists}. This is known as existential quantification. The dual problem, universal quantification, is to determine whether all variable assignments satisfy a formula. Both forms of quantification are required for synthesis in order to decide whether an action exists for one player that satisfies a property for all opponent actions. An example of this kind of computation would be deciding whether the controller can force the game into the winning region regardless of any action the environment chooses. It is possible to perform universal quantification with a SAT solver but it adds considerable complexity, which introduces another bottleneck to the synthesis process.

\section{Approach}

This thesis presents a SAT based approach that computes an approximation of the winning region. By approximating the winning region we hope to avoid the state explosion cost of representing the entire set of winning states. The algorithm is set within a counterexample guided abstraction refinement framework. This is our approach to handling the alternating quantifications of synthesis. Candidate strategies are constructed and refined via counterexamples instead of precisely computing the result of the quantified formula.

In this approach, a SAT solver is used to verify whether a candidate strategy is a winning strategy for a safety game with a fixed number of a game rounds, which we call a bounded game. This approach is similar to bounded model checking where a program is verified by querying a SAT solver for a trace that violates the specification. In our bounded synthesis approach the SAT solver searches for a trace of opponent moves that cause the candidate strategy to lose the bounded game. As with bounded model checking, a counterexample trace informs the algorithm how to refine the candidate strategy.

Discovering a winning strategy for the bounded game does not guarantee that the strategy is winning in the unbounded game. Specifically, if the controller strategy avoids an error state for $k$ rounds the SAT solver cannot guarantee that it can avoid errors for $k+1$ rounds. We address this problem with an extension to the algorithm that iteratively solves bounded games while incrementing the bound. During the execution of the bounded game solver we learn losing states for both players. This computational learning serves a dual purpose by both serving as an optimisation to reduce the search space and also providing the termination condition. By carefully learning states that are losing for the environment we may construct an overapproximation of the environment's winning region. The overapproximation can be used to guarantee that the actual winning region does not contain the initial states and so there cannot be a winning strategy for the environment.


\section{Contribution}

This thesis presents a SAT based counterexample guided approach to controller synthesis of safety specifications. This approach includes a bounded synthesis algorithm, an extension to unbounded synthesis, and a methodology for extracting strategies from the certificate generated by the bounded synthesis algorithm. 

The approach is designed to solve synthesis specifications where the winning region is difficult to represent compactly with existing symbolic techniques. The aim of this work was not to produce a one size fits all approach to safety synthesis but instead to provide a solution suited to problem instances that are difficult to solve for other methods.

The instances that emit winning regions that are difficult to efficiently represent with binary decision diagrams include many real world problems. An example of such a specification is an arbiter that must grant resources from a homogeneous pool in order to fulfil requests from the environment. In this problem the winning region for the environment must exclude all combinations of resource allocations that exceed the number of requests. There is no compact representation of this kind of winning region as a binary decision diagram but in our approach we use overapproximation so that we don't need to find and represent all combinations.

In order to validate the methodology I have implemented the algorithm as an open source tool. In later chapters we present benchmarks that show that the algorithm is promising and although it does not solve as many problem instances as other techniques it performs better on certain classes of problems.


\section{Summary}

\begin{itemize}

    \item \emph{Reactive synthesis} can be used to automatically generate correct-by-construction controllers for software systems. Compared to other approaches to software correctness synthesis does not require the software to first be developed.

    \item Synthesis is formalised as a game between a controller and its environment. In many cases these games can be solved by constructing a symbolic representation of the winning states of the game using a binary decision diagram. However, for some games there is no compact representation of the winning region.

    \item This thesis presents a SAT based counterexample guided approach that targets these cases by constructing an approximation of the winning region that is sufficient to determine the winner of the game.

\end{itemize}


%%%\section{Device Drivers}

%%%Device drivers are the software that allows the operating system to interface
%%%with hardware. The role of the driver is to manipulate the inputs of the device
%%%so that it remains in a error-free state and correctly handles the requests of
%%%the operating system. By way of example consider an ethernet driver. The driver
%%%accepts requests to send and receive data packets from the OS and acts on those
%%%requests by reading from and writing to buffers on the device. It must ensure
%%%that those buffers are maintained in a usable state by correctly updating a
%%%register containing the location of the head of the buffer queue.

%%%According to a study performed in 2011~\cite{Palix11}, drivers account for
%%%approximately 57\% of the lines of code in the Linux kernel and subsequently is
%%%the largest source of bugs. The study also analysed the staging directory of
%%%the kernel, which contains all in-progress drivers, and found it to contain the
%%%highest fault rate (faults per line of code) out of any directory in the
%%%kernel. The results of the study give evidence to the widely held belief that
%%%correct drivers are hard to produce.

%%%Consequences of buggy drivers

%%%This thesis focuses on automatic construction of correct drivers as a solution
%%%to the driver problem. Alternate approaches, of which there are many, will be
%%%discussed in Chapter~\ref{ch:relatedwork}.

