\chapter{Related Work}
\label{ch:relatedwork}

Reactive synthesis is an extensively studied topic and the work of this thesis is influenced by a wide array of prior work. In the previous chapter we identified symbolic representation of state sets and abstraction refinement as methodologies for mitigating state explosion. In this chapter we will approach the problem from a different angle. The work in this thesis is inspired by research in the model checking community and some prior efforts to apply that research to synthesis.

\section{Bounded model checking}

Bounded model checking~\cite{Biere99} (BMC), as introduced in Chapter~\ref{sec:boundedmodelchecking}, is a methodology that generates SAT queries to determine the existence of a trace in a game that violates its specification. The approach taken to realisability described in Chapter~\ref{ch:bounded} is inspired by BMC and also unrolls the transition relationship into a SAT query and searches for counterexample traces. 

BMC considers the validity of CTL* formulas in Kripke structures in which the game is bounded to $k$ rounds. The authors of the procedure provide a semantics for the translation of CTL* formulas on bounded models, via LTL, to satisfiability constraints. Consider a safety property $AG \phi$ as an example. A safety property is universal and so is checked in BMC by searching existentially for counterexamples in the form of the negation $F \lnot \phi$. The search is translated into a SAT query by unrolling the transition relation $R$ like so: $s_0 \land \bigwedge_{i=0}^{k-1} R(s_i, s_{i+1})$. The LTL formula is similarly translated into a formula $\bigvee_{i=0}^{k} \lnot \phi \in L(s_i)$. The SAT query is equivalent to a check whether there is a path in the bounded Kripke structure $(s_0, s_1, ..., s_k)$ such that $\lnot \phi$ holds in some state $s_i$.

For bounded realisability, the unrolling of the transition relation and the translation of the safety property can be done in much the same way. The difference is that in a synthesis problem the model has yet to be constructed and the algorithm must be given the freedom to choose the actions of the controller. For more details about how this can be overcome, see Chapter~\ref{ch:bounded}.

One of the motivations behind bounded model checking is aligned with the aim of this thesis: to avoid the high cost in space of approaches that construct a symbolic representation of the winning region as a binary decision diagram. By bounding the length the game the procedure can rely on SAT as a symbolic representation instead of BDDs. The drawback is that although BMC is sound, any counterexample is a true counterexample, it is not complete with respect to the unbounded game unless a sufficient bound is used. For safety properties the diameter of the game gives a tight sufficient upper bound for BMC although it is difficult to compute. Due to the additional complexity of synthesis it is not feasible to use the diameter in a bounded synthesis procedure.

\section{Unbounded model checking}

The usage of SAT solvers in bounded model checking proved to be highly beneficial for discovering counterexamples. Research into applications of SAT in unbounded model checking has subsequently progressed in several directions.

\subsection{Non-canonical symbolic representation}

One hybrid approach to unbounded model checking is to replace BDDs with binary expression diagrams (BEDs) or reduced boolean circuits (RBCs) in a fixed point algorithm~\cite{Williams00, Abdulla00}. BEDs are a generalisation of BDDS with the advantage that BEDs are not canonical and their use as a symbolic representation may be more succinct than the equivalent BDD. An RBC is simply a graphical representation of a circuit with some reductions applied. The two representations are essentially orthogonal and conversion between them is linear.

The disadvantage is that the controllable predecessor computations during the fixed point calculation require quantifier elimination that increase the size of the BED or RBC. Detecting a fixed point in the state sets then requires a costly satisfiability check of combinations of the expanded formulas. One option is to construct an equivalent BDD for which the satisfiability check is efficient but potentially negates the advantage of the non-canonical representation. It is also possible to construct a CNF representation of the formula from either a BED or RBC and query a SAT solver for satisfiability. Neither option fully mitigates the potential size of the expanded formula due to quantifier elimination. In practice this methodology works well only on models with few inputs so that quantification does not explode the formulas.


