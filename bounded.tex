\chapter{Bounded Realisability}
\label{ch:bounded}

\newtheorem*{exmp}{Example}
\newtheorem*{exmpI}{Example: Intuition behind the algorithm}

In this chapter I will describe my work on bounded realisability of reactive systems with safety properties. As introduced in Chapter~\ref{ch:background} reactive realisability is the problem of determining the existence of a program, which we call a \emph{controller}, that continuously interacts with its environment in adherence with a specification. A safety property is a simple correctness condition that lays out a set of states of the system that controller must stay within. In this chapter we will refer to this property in the negation: the controller must avoid \emph{error states}.

Realisability is the first step on the path to synthesis. In the subsequent chapter I will describe an algorithm that extracts the actions of the controller necessary for realisation. This strategy may be used for synthesis: automatic construction of the controller program. Reactive synthesis for controllers with safety properties has many practical uses in areas such as circuit design, device drivers, or industrial automation.

The algorithm described in this chapter solves bounded safety games. Recall that Chapter~\ref{ch:background} introduced games as a formalism for synthesis by stating the problem in terms of a game between a controller and its environment. In this chapter we are concerned with \emph{bounded} games that restrict all runs in the game to certain length. This concept is borrowed from model checking where it is used to verify that a program emits no erroneous traces of a certain length. A propositional formula may be constructed that is satisfiable when a trace that visits an error state exists. A SAT solver can be used to efficiently search for a satisfying assignment to this formula, which represents a counterexample to the correctness property of the specification. 

In realisability it is not enough to check for the existence of a trace that visits an error state. Such a trace only implies a counterexample that requires the controller to cooperate with the environment to fail. Instead we are interested in strategies for the players. A controller strategy must avoid the error states for all possible environment actions. Likewise, a counterexample strategy must take into account all controller actions. We cannot use a SAT solver to search for a strategy directly as now we require quantifiers. We can, however, check if a strategy allows a counterexample trace without quantification. This sets us up for a counterexample guided methodology in which we construct candidate strategies and check them for correctness. If we discover a counterexample we use it to guide a refinement step in which we improve the candidate strategy.

Similar to bounded model checking, bounded realisability is not a complete procedure.  If we decide that the controller can avoid error states for a game bounded to $k$ rounds there is no guarantee that the environment can not force an error in a game with a bound higher than $k$. In Chapter~\ref{ch:unbounded} I present an extension to the algorithm that extends this algorithm to unbounded games.

Restricting ourselves to solving a bounded safety game enables us to turn the focus of the algorithm from states to traces. The traditional approach of constructing a binary decision diagram to symbolically represent the winning region has the potential to consume exponential space. The advantage of concentrating on runs of the game is that we do not rely on computing the winning states and therefore do not suffer from the related state explosion. The factors affecting the upper limit on scalability for the bounded synthesis algorithm are different to those of the BDD based approach. The most efficient algorithm for a realisability problem depends on the properties of that problem instance.

\section{Algorithm}

This work draws inspiration from a QBF solving algorithm that treats the QBF problem as a game~\cite{Janota12}. In that algorithm one player assumes the role of the universal quantifiers and the opponent takes on the existential quantifiers. In the game, the players take turns to chooses values for their variables from the outermost quantifier block in. Quantifiers may be removed from a formula by iteratively constructing and merging copies of the formula for each quantified variable. The copies of the formula represent the two possible values, true or false, of the quantified boolean variable. Universal quantification can then be reduced to the conjunction of these copies, and existential quantifications corresponds to a disjunction. In practice, these expanded formulas are far too large to be solved so the authors introduce abstractions, or partially expanded formulas, to avoid expanding on variables unnecessarily. The abstractions are refined through a CEGAR process of searching for candidate solutions and analysing counterexamples. The full algorithm is described in detail in Chapter~\ref{ch:related}. 

A quantified formula may be used to solve realisability of a bounded synthesis game. We construct the formula by unrolling the transition relationship for every game round up until the bound. A quantifier alternation is introduced to the formula for the variables corresponding to the actions of each player. Universal quantifiers are used for the environment variables and existential for the controller. The formula is constrained so that if the state at any game round is an error state the formula evaluates to false. In this way the formula is a satisfiable if and only if a strategy exists for the controller that avoids the error states.

It is possible to solve this QBF na\"ively but we can do better by taking into account the inherent structure that arises in formulas constructed from realisability problems. The games we solve are deterministic so we know that state variables at a particular game unrolling are fully dependent on the action variables that come before. This enables a fine grained computational learning optimisation that would be difficult to reproduce in a general QBF algorithm.

The bounded realisability algorithm presented here takes inspiration from the CEGAR framework of the work of Janota et. al. and can be thought of as a domain specific version of the QBF algorithm.

\section{Intuition}

We introduce an example to assist an intuitive explanation of the algorithm. Consider a simple model of an ethernet device driver. The operating system makes requests of the driver to write or read data to or from the device. It is the role of the driver to grant these requests while ensuring that a \texttt{read} never occurs when a \texttt{write} was requests and vice versa.

As detailed in Chapter~\ref{ch:background}, we formalise realisability by a game structure $G = (S, \mathcal{U}, \mathcal{C}, \delta, s_0)$. The structure for our example is:

\begin{itemize}
    \item $S = \{ \texttt{request}, \texttt{error} \} $. The game consists of two boolean variables to denote the current request, and whether an error has occurred. We use \texttt{request = 0} to represent a \texttt{read} and \texttt{request = 1} for \texttt{write}.
    \item $\mathcal{U} = \{ \texttt{os\_request} \} $. The uncontrollable actions consists of a single boolean variable to describe a \texttt{read} or a \texttt{write}. We use the same values as before, 0 for \texttt{read} and 1 for \texttt{write}.
    \item $\mathcal{C} = \{ \texttt{dev\_cmd} \}$. The controllable actions similarly consists of a single boolean variable to denote the command given to the device: a \texttt{read} (\texttt{dev\_cmd = 0}) or a \texttt{write} (\texttt{dev\_cmd = 1}).
    \item The transition relation $\delta$ is defined by the following formula:

        $$ \texttt{request'} \gets \texttt{os\_request} $$

        $$ \texttt{err'} \gets \texttt{request} \neq \texttt{dev\_cmd} $$

        Primed variables are used here to indicate how the value is assigned in the next game round.

    \item $s_0 = \{ \texttt{request = 0, err = 0} \}$. To simplify the example there is no idle state so the model is initialised with a pending \texttt{read} request.

\end{itemize}

\begin{figure}
    \centering
    \begin{tikzpicture}[level distance = 5mm,baseline]
            \node [circle,draw] (n00){(0, 0)};
            \node [circle,draw,right=of n00] (n10){(1, 0)};
            \node [circle,draw,below=of n00] (n01){(0, 1)};
            \node [circle,draw,below=of n10] (n11){(1, 1)};

            \draw (n00) -- (n01);
    \end{tikzpicture}
    \label{fig:example1}
    \caption{State automata representation of $\delta$ in Example 1.}
\end{figure}

Our bounded synthesis algorithm is set within a counterexample guided abstraction refinement framework. Abstraction serves a dual purpose in our approach as both a method for discovering player strategies and as a way to reduce the search space of the game. This is achieved by employing one player's candidate strategy as its opponent's game abstraction. The effect is that the search for a player's strategy is directed by its opponent's current best effort at its own strategy and the search is guided toward winning strategies.

We will now step through an execution of the algorithm using the example we have just introduced. The first step involves a search of the empty game abstraction for an initial candidate strategy for the environment player. In the empty abstraction we have not yet restricted the game in any way so all runs through the game are enabled. We search for a candidate strategy by finding a run that reaches an error state. Here we are only searching for the existence of a run and so we do not require quantifier alternations. A SAT solver can be used to efficiently do this search for us. Intuitively, an existential search is equivalent to the two players of the game cooperating. In most real world applications of synthesis the environment is in fact cooperating with the system so the cooperation heuristic turns out to be practical. Figure~\ref{fig:example1a} shows a trace through the example game that reaches an error state.

Our bounded synthesis algorithm constructs abstractions of the game that restrict actions available to one of the players.  Specifically, we consider abstractions represented as trees of actions, referred to as \emph{abstract game trees} (AGTs).  Figure~\ref{fig:agt} shows an example abstract game tree restricting the environment (abstract game trees restricting the controller are similar).  In the abstract game, the controller can freely choose actions whilst the environment is required to pick actions from the tree.  After reaching a leaf, the environment continues playing unrestricted.  The tree in Figure~\ref{fig:agt} restricts the first environment action to \texttt{request=1}. At the leaf of the tree the game continues unrestricted.

\section{Case Study: Arbiter}

We consider a simple arbiter system in which the environment makes a request for a number of resources (1 or 2), and the controller may grant access to up to two resources.  The total number of requests grows each round by the number of environment requests and shrinks by the number of resources granted by the controller in the previous round.  The controller must ensure that the number of unhandled requests does not accumulate to more than 2.  Figure~\ref{fig:example} shows the variables (\ref{fig:examplevars}), the initial state of the system (\ref{fig:exampleinit}), and the formulas for computing next-state variable assignments (\ref{fig:exampletrans}) for this example. We use primed identifiers to denote next-state variables and curly braces to define the domain of a variable.

This example is the $n=2$ instance of the more general problem of an arbiter of $n$ resources. For large values of $n$, the set of winning states has no compact representation, which makes the problem hard for BDD solvers. In Section~3 we will outline how the unbounded game can be solved without enumerating all winning states.

\begin{figure}
    \begin{subfigure}[t]{\textwidth}
        \centering
        \begin{tabular}{l | l | l}
            \textbf{Controllable} & \textbf{Uncontrollable} & \textbf{State} \\
            \hline
            \texttt{request : \{1, 2\}} & \texttt{grant0 = \{0, 1\}} & \texttt{resource0 = \{0, 1\}} \\
            & \texttt{grant1 : \{0, 1\}} & \texttt{resource1 = \{0, 1\}} \\
            & & \texttt{nrequests : \{0, 1, 2, 3\}} \\
        \end{tabular}
        \caption{Variables}
        \label{fig:examplevars}
    \end{subfigure}

    \begin{subfigure}[t]{\textwidth}
        \centering
        \texttt{resource0 = 0; resource1 = 0; nrequests = 0;}
        \caption{Initial State}
        \label{fig:exampleinit}
    \end{subfigure}

    \begin{subfigure}[t]{\textwidth}
        \begin {align*}
            \texttt{resource0'} & \texttt{ = grant0;} \\
            \texttt{resource1'} & \texttt{ = grant1;} \\
            \texttt{nrequests'} & \texttt{ = (nrequests + request >= resource0 + resource1)} \\ 
                                & \texttt{ ? (nrequests + request - resource0 - resource1) : 0;}
        \end{align*}
        \caption{Transition Relation}
        \label{fig:exampletrans}
    \end{subfigure}
    \caption{Example}
    \label{fig:example}
\end{figure}

Our bounded synthesis algorithm constructs abstractions of the game that restrict actions available to one of the players.  Specifically, we consider abstractions represented as trees of actions, referred to as \emph{abstract game trees} (AGTs).  Figure~\ref{fig:agt} shows an example abstract game tree restricting the environment (abstract game trees restricting the controller are similar).  In the abstract game, the controller can freely choose actions whilst the environment is required to pick actions from the tree.  After reaching a leaf, the environment continues playing unrestricted.  The tree in Figure~\ref{fig:agt} restricts the first environment action to \texttt{request=1}. At the leaf of the tree the game continues unrestricted.

The root of the tree is annotated by the initial state $s$ of the abstract game
and the bound $k$ on the number of rounds.  We denote $\textsc{nodes}(T)$ the
set of all nodes of a tree $T$, $\textsc{leaves}(T)$ the subset of leaf nodes.
For edge $e$, $\textsc{action}(e)$ is the action that labels the edge, and for
node $n$, $\textsc{height}(k, n)$ is the distance from n to the last round of a
game bounded to $k$ rounds.  $\textsc{height}(k, T)$ is the height of the root
node of the tree.  For node $n$ of the tree, $\textsc{succ}(n)$ is the set of
pairs $\langle e, n' \rangle$ where $n'$ is a child node of $n$ and $e$ is the
edge connecting $n$ and $n'$.

\tikzset{every node/.style={solid}}
\tikzstyle{fixed}=[solid]
\begin{figure}
    \centering
    \captionsetup[subfigure]{width=\textwidth,justification=raggedleft}
    \begin{subfigure}[t]{.4\textwidth}
        \centering
        \begin{minipage}[t][3.8cm][t]{\textwidth}
        \begin{tikzpicture}[level distance = 5mm,baseline]
            \node [circle,draw,inner sep=1pt] (root){}
                child {node [circle,draw,inner sep=1pt] {}
                    child {node [circle,draw,inner sep=1pt] {}
                        child {node [circle,draw,inner sep=1pt] {}
                            child {node [circle,draw,inner sep=1pt] {}
                                child {node [circle,draw,inner sep=1pt] {}
                                    node [left=4pt] {\texttt{gr0=0 gr1=0}}
                                }
                                node [left=4pt] {\texttt{req=1}}
                            }
                            node [left=4pt] {\texttt{gr0=0 gr1=0}}
                        }
                        node [left=4pt] {\texttt{req=1}}
                    }
                node [left=4pt] {\texttt{gr0=1 gr1=0}}
                }
                node [left=4pt] {\texttt{req=1}}
                node [above=4pt] {$\langle s, 3 \rangle$};
        \end{tikzpicture}
    \end{minipage}
        \caption{Controller winning trace}
        \label{fig:trace}
    \end{subfigure}
    \begin{subfigure}[t]{.4\textwidth}
        \centering
        \begin{minipage}[t][3.8cm][t]{\textwidth}
        \begin{tikzpicture}[dash pattern = on 2pt off 2pt, level distance = 10mm,baseline]
            \node [circle,draw] (root){}
                child {node [circle,draw] {}
                    edge from parent [fixed] node [left] {\texttt{gr0=1 gr1=0}}
                }
                node [left=4pt] {}
                node [above=4pt] {$\langle s, 3 \rangle$};
        \end{tikzpicture}
        \end{minipage}
        \caption{AGT}
        \label{fig:agt}
    \end{subfigure}

    \begin{subfigure}[t]{.4\textwidth}
        \centering
        \begin{minipage}[t][3.8cm][t]{\textwidth}
        \begin{tikzpicture}[level distance = 5mm,baseline]
            \node [circle,draw,inner sep=1pt] (root){}
                child {node [circle,draw,inner sep=1pt] {}
                    child {node [circle,draw,inner sep=1pt] {}
                        child {node [circle,draw,inner sep=1pt] {}
                            child {node [circle,draw,inner sep=1pt] {}
                                child {node [circle,draw,inner sep=1pt] {}
                                    node [left=4pt] {\texttt{gr0=0 gr1=0}}
                                }
                                node [left=4pt] {\texttt{req=1}}
                            }
                            node [left=4pt] {\texttt{gr0=0 gr1=0}}
                        }
                        node [left=4pt] {\texttt{req=1}}
                    }
                node [left=4pt] {\texttt{gr0=1 gr1=0}}
                }
                node [left=4pt] {\texttt{req=2}}
                node [above=4pt] {$\langle s, 3 \rangle$};
        \end{tikzpicture}
        \end{minipage}
        \caption{Environment winning trace}
        \label{fig:trace2}
    \end{subfigure}
    \begin{subfigure}[t]{.4\textwidth}
        \centering
        \begin{minipage}[t][3.8cm][t]{\textwidth}
        \begin{tikzpicture}[dash pattern = on 2pt off 2pt, level distance = 10mm,baseline]
            \node [circle,draw] (root){}
                child {node [circle,draw] {}
                    edge from parent [fixed] node [left] {\texttt{gr0=1 gr1=0}}
                }
                node [left=4pt] {\texttt{req=2}}
                node [above=4pt] {$\langle s, 3 \rangle$};
        \end{tikzpicture}
        \end{minipage}
        \caption{Partial Strategy}
        \label{fig:strategy}
    \end{subfigure}
    \caption{Abstract game trees.}
    \label{fig:alltrees}
\end{figure}


Given an environment (controller) abstract game tree $T$ a \emph{partial
strategy} $Strat: \textsc{nodes}(T) \rightarrow \mathcal{C}$ ($Strat: \textsc{nodes}(T)
\rightarrow \mathcal{U}$) labels each node of the tree with the controller's
(environment's) action to be played in that node.   Given a partial strategy
$Strat$, we can map each leaf $l$ of the abstract game tree to $\langle
s',i'\rangle=\textsc{outcome}(\langle s, i\rangle, Strat, l)$ obtained by
playing all controllable and uncontrollable actions on the path from the root
to the leaf.  An environment (controller) partial strategy is \emph{winning against $T$} 
if all its outcomes are states that are winning for the environment (controller)
in the concrete game.


%%%Figure~\ref{fig:strategy} shows an example partial strategy for
%%%the controller.  The controller responds to the requests by first granting
%%%access to \texttt{resource1}, then to \texttt{resource0} in the second round.

\begin{exmpI}

    %% Make it clear that this is intuition only
    We present the intuition behind our bounded synthesis method by applying
    its \emph{simplified version} to the running example.  We begin by finding
    a trace of length $k$ (here we consider $k=3$) that is winning for the
    controller, i.e., that starts from the initial state and avoids the error
    set for three game rounds (see Figure~\ref{fig:trace}).  We use a SAT
    solver to find such a trace, precisely as one would do in bounded model
    checking.  Given this trace we make an initial conjecture that any trace
    starting with action \texttt{gr0=1 gr1=0} is winning for the controller.
    This conjecture is captured in the abstract game tree shown in
    Figure~\ref{fig:agt}.  We validate this conjecture by searching for a
    counterexample trace that reaches an error state with the first controller
    action fixed to \texttt{gr0=1 gr1=0}.   Such a trace, that refutes the
    conjecture, is shown in Figure~\ref{fig:trace2}.  In this trace, the
    environment wins by playing \texttt{req=2} in the first round.  This move
    represents the environment's partial strategy against the abstract game
    tree in Figure~\ref{fig:agt}.  This partial strategy is shown in
    Figure~\ref{fig:strategy}.
    
    Next we strengthen the abstract game tree taking this partial strategy into account.
    To this end we again use a SAT solver to find a trace where the contoller
    wins while the environment plays according to the partial strategy.  In the
    resulting trace (Figure~\ref{fig:trace3}), the controller plays \texttt{gr0=1 gr1=1} in
    the second round.  We refine the abstract game tree using this move as
    shown in Figure~\ref{fig:refined1}.  The environment's partial strategy was
    to make two requests in the first round, to which the controller responds
    by now granting an additional two resources in the second round.

    When the controller cannot refine the tree by extending existing branches,
    it backtracks and creates new branches. Eventually, we obtain the abstract
    game tree shown in Figure~\ref{fig:refined2} for which there does not exist
    a winning partial strategy on behalf of the environment.  We conclude that
    the bounded game is winning for the controller.

\end{exmpI}

\begin{figure}
    \centering
    \begin{subfigure}[t]{.2\textwidth}
        \centering
        \begin{minipage}[t][3.8cm][t]{\textwidth}
        \begin{tikzpicture}[level distance = 5mm,baseline]
            \node [circle,draw,inner sep=1pt] (root){}
                child {node [circle,draw,inner sep=1pt] {}
                    child {node [circle,draw,inner sep=1pt] {}
                        child {node [circle,draw,inner sep=1pt] {}
                            child {node [circle,draw,inner sep=1pt] {}
                                child {node [circle,draw,inner sep=1pt] {}
                                    node [left=4pt] {\texttt{gr0=0 gr1=0}}
                                }
                                node [left=4pt] {\texttt{req=1}}
                            }
                            node [left=4pt] {\texttt{gr0=1 gr1=1}}
                        }
                        node [left=4pt] {\texttt{req=1}}
                    }
                node [left=4pt] {\texttt{gr0=1 gr1=0}}
                }
                node [left=4pt] {\texttt{req=2}}
                node [above=4pt] {$\langle s, 3 \rangle$};
        \end{tikzpicture}
        \end{minipage}
        \caption{Controller winning trace}
        \label{fig:trace3}
    \end{subfigure}
    \begin{subfigure}[t]{.3\textwidth}
        \centering
        \begin{minipage}[t][3.8cm][t]{\textwidth}
        \hspace*{0.8cm}
        \begin{tikzpicture}[dash pattern = on 2pt off 2pt, level distance = 10mm,baseline]
            \node [circle,draw] (root){}
                child {node [circle,draw] {}
                    child {node [circle,draw] {}
                        edge from parent [fixed] node [left] {\texttt{gr0=1 gr1=1}}
                    }
                    edge from parent [fixed] node [left] {\texttt{gr0=1 gr1=0}}
                }
                node [above=4pt] {$\langle s, 3 \rangle$};
        \end{tikzpicture}
        \end{minipage}
        \caption{First refined AGT}
        \label{fig:refined1}
    \end{subfigure}
    \begin{subfigure}[t]{.4\textwidth}
        \centering
        \begin{minipage}[t][3.8cm][t]{\textwidth}
        \begin{tikzpicture}[dash pattern = on 2pt off 2pt, level distance = 10mm,baseline]
            \node [circle,draw] (root){}
                child {node [circle,draw] {}
                    child {node [circle,draw] {}
                        child {node [circle,draw] {}
                            edge from parent [fixed] node [left] {\texttt{gr0=1 gr1=1}}
                        }
                        edge from parent [fixed] node [left] {\texttt{gr0=1 gr1=1}}
                    }
                    edge from parent [fixed] node [left] {\texttt{gr0=1 gr1=0}}
                }
                child {node [circle,draw] {}
                    child {node [circle,draw] {}
                        child {node [circle,draw] {}
                            edge from parent [fixed] node [right] {\texttt{gr0=1 gr1=1}}
                        }
                        edge from parent [fixed] node [right] {\texttt{gr0=1 gr1=1}}
                    }
                    edge from parent [fixed] node [right] {\texttt{gr0=1 gr1=1}}
                }
                node [above=4pt] {$\langle s, 3 \rangle$};
        \end{tikzpicture}
        \end{minipage}
        \caption{Final Refined AGT}
        \label{fig:refined2}
    \end{subfigure}
    \caption{Refined abstract game trees.}
    \label{fig:refinedtrees}
\end{figure}


\begin{algorithm}
    \begin{algorithmic}[1]
        \Function{solveAbstract}{$p, s, k, T$}
        \State $cand \gets $ \Call{findCandidate}{$p, s, k, T$} \Comment{Look for a candidate}
        \IIf{$k = 1$} \Return $cand$ \EndIIf \Comment{Reached the bound}
        \State $T' \gets T$
        \Loop
            \IIf{$cand = \texttt{NULL}$} \Return $\texttt{NULL}$ \EndIIf \Comment{No candidate: return with no solution}
            \State $\langle cex, l, u \rangle \gets $ \Call{verify}{$p, s, k, T, cand$} \Comment{Verify candidate}
            \IIf{$cex = \False$} \Return $cand$ \EndIIf \Comment{No counterexample: return candidate}
            \State $T' \gets $ \Call{append}{$T', l, u$} \Comment{Refine $T'$ with counterexample}
            \State $cand \gets $ \Call{solveAbstract}{$p, s, k, T'$} \Comment{Solve refined game tree}
        \EndLoop
        \EndFunction
        \algstore{b1}
    \end{algorithmic}

    \begin{algorithmic}
        \algrestore{b1}
        \Function{findCandidate}{$p, s, k, T$}
        \State $\hat{T} \gets $ \Call{extend}{$T$} \Comment{Extend the tree with unfixed actions}
            \State $f \gets $ \IfElse{$p = \texttt{cont}$}{\Call{treeFormula}{$k, \hat{T}$}}{\Call{\textoverline{treeFormula}}{$k, \hat{T}$}} \EndIfElse
            \State $sol \gets $ \Call{SAT}{$s(X_{\hat{T}}) \land f$}
            \If{$sol = \texttt{unsat}$} 
                \If{\texttt{unbounded}} \Comment{Active only in the unbounded solver}
                    %\State $\sigma \gets $ \Call{generalise}{$s$} \Comment{Expand $s$ to a set of states}
                    \State \IfElse{$p = \texttt{cont}$}{\Call{learn}{$s, \hat{T}$}}{\Call{\textoverline{learn}}{$s, \hat{T}$}} \EndIfElse
                \EndIf
                \State \Return $\texttt{NULL}$ \Comment{No candidate exists}
            \Else
                \State \Return $\{ \langle n, c \rangle | n \in $ \Call{nodes}{$T$} $, c = \Call{sol}{n} \}$ \Comment{Fix candidate moves in $T$}
            \EndIf
        \EndFunction
        \algstore{b2}
    \end{algorithmic}

    \begin{algorithmic}
        \algrestore{b2}
        \Function{verify}{$p, s, k, T, cand$}
            \For{$l \in leaves(gt)$}
            \State $\langle k', s'\rangle \gets $ \Call{outcome}{$s, k, cand, l$} \Comment{Get bound and state at leaf}
                \State $u \gets $ \Call{solveAbstract}{\Call{opponent}{$p$}, $s'$, $k'$, $\emptyset$} \Comment{Solve for the opponent}
                \IIf{$u \neq \texttt{NULL}$} \Return $\langle \True, l, u \rangle$ \EndIIf \Comment{Return counterexample}
            \EndFor
            \State \Return $\langle \False, \emptyset, \emptyset \rangle$
        \EndFunction
    \end{algorithmic}

    \caption{Bounded synthesis}
    \label{alg:bounded}
\end{algorithm}

The full bounded synthesis algorithm is more complicated: upon finding a candidate 
partial strategy on behalf of player $p$ against abstract game tree $T$, it first checks whether the strategy is winning 
against $T$.  By only considering such strong candidates, we reduce the number of 
refinements needed to solve the game.  To this end, the algorithm checks whether each 
outcome of the candidate strategy is a winning state for $\textsc{opponent}(p)$ by recursively 
invoking the synthesis algorithm on behalf of the opponent.  Thus, our bounded
synthesis algorithm can be seen as running two competing solvers, for the
controller and for the environment. 

%The solvers build candidate partial strategy 
%for their corresponding players, which are used to refine the abstractions.


The full procedure is illustrated in Algorithm~\ref{alg:bounded}.  The
algorithm takes a concrete game $G$ with maximum bound $\kappa$ as an implicit
argument.  In addition, it takes a player $p$ (controller or environment),
state $s$, bound $k$ and an abstract game tree $T$ and returns a winning
partial strategy for $p$, if one exists.  The initial invocation of the
algorithm takes the initial state $I$, bound $\kappa$ and an empty abstract
game tree $\emptyset$.  Initially the solver is playing on behalf of the
environment since that player takes the first move in every game round.  The
empty game tree does not constrain opponent moves, hence solving such an
abstraction is equivalent to solving the original concrete game.

The algorithm is organised as a counterexample-guided abstraction refinement
(CEGAR) loop.  The first step of the algorithm uses the \textsc{findCandidate}
function, described below, to come up with a candidate partial strategy that is
winning when the opponent is restricted to $T$.  If it fails to find a
strategy, this means that no winning partial strategy exists against the opponent
playing according to $T$.  If, on the other hand, a
candidate partial strategy is found, we need to verify if it is indeed winning
for the abstract game $T$.

The \textsc{verify} procedure searches for a \emph{spoiling} counterexample
strategy in each leaf of the candidate partial strategy by calling
\textsc{solveAbstract} for the opponent. The dual solver solves games on behalf
of the opponent player.  

If the dual solver can find no spoiling strategy at any of the leaves, then the
candidate partial strategy is a winning one. Otherwise, \textsc{verify} returns
the move used by the opponent to defeat a leaf of the partial strategy, which
is appended to the corresponding node in $T$ in order to refine it in line~(9).

We solve the refined game by recursively invoking \textsc{solveAbstract} on it.
If no partial winning strategy is found for the refined game then there is also
no partial winning strategy for the original abstract game, and the algorithm
returns a failure.  Otherwise, the partial strategy for the refined game is
\emph{projected} on the original abstract game by removing the leaves
introduced by refinements. The resulting partial strategy becomes a candidate
strategy to be verified at the next iteration of the loop. In the worst case
the loop terminates after all actions in the game are refined into the abstract
game.

The CEGAR loop depends on the ability to guess candidate partial strategies in
\textsc{findCandidate}. For this purpose we use the heuristic that a partial
strategy may be winning if each \textsc{outcome} of the strategy can be
extended to a run of the game that is winning for the current player.  Clearly,
if such a partial strategy does not exist then no winning partial strategy can
exist for the abstract game tree. We can formulate this heuristic as a SAT
query, which is constructed recursively by $\textsc{treeFormula}$ (for the
controller) or $\textsc{\textoverline{treeFormula}}$ (for the environment) in
Algorithm~\ref{alg:treeFormula}.

The tree is first extended to the maximum bound with edges that are labeled
with arbitrary opponent actions (Algorithm~\ref{alg:bounded}, line 14).  For
each node in the tree, new SAT variables are introduced corresponding to the
state ($X_T$) and action ($U_T$ or $C_T$) variables of that node. Additional
variables for the opponent actions in the edges of $T$ are introduced ($U_e$ or
$C_e$) and set to $\textsc{action}(e)$.  The state and action variables of node
$n$ are connected to successor nodes $\textsc{succ}(n)$ by an encoding of the
transition relation and constrained to the winning condition of the player.

%%%Then copies of state and action variables are introduced for each node in the
%%%tree and opponent action variables for each edge $e$ are set to
%%%$\textsc{action}(e)$. 

%%%Candidates are discovered by passing a formulation of the abstract game tree to
%%%a SAT solver in $\textsc{findCandidate}$. This formula contains CNF encodings
%%%of all of the unrolled runs represented by the tree and the winning condition
%%%of the current player.  Runs are encoded by copying the transition relation for
%%%every step in the abstract game. When playing for the controller, the SAT
%%%solver searches for a satisfying assignment to the unfixed label variables in
%%%tree so that none of the runs reaches the error state. The environment
%%%formulation is satisfiable if any run does reach the error state.  The formula
%%%is constructed recursively from the root of a tree by $\textsc{treeFormula}$
%%%(see Algorithm~\ref{alg:treeFormula}).

%%%Since the game tree formulation is passed to a SAT solver, both controllable
%%%and uncontrollable unfixed labels will be existentially quantified. This means
%%%that the SAT solver will find any way to win the game while both players are
%%%cooperating. If no winning run exists in an abstract game even when the players
%%%are cooperating then there is no winning run when the opponent is playing
%%%adversarily. When a winning run is found, the actions chosen by the SAT solver
%%%are used to refine the game tree. This is advantageous for many synthesis
%%%problems where the game must be formalised as adversarial for correctness but
%%%the final implementation will cooperate with its environment in the real world.
%%%An example of such a system is a device driver that cooperates with the device
%%%and OS to provide the interface between the two.

\begin{algorithm}
    \caption{Tree formulas for Controller and Environment respectively}
    \label{alg:treeFormula}
    \begin{algorithmic}[1]
        \Function{treeFormula}{$k, T$}
        \If{$\Call{height}{k, T} = 0$}
        \State \Return{ $\lnot \Call{E}{X_{T}}$ }
        \Else
        \State \Return{$\lnot \Call{E}{X_{T}} \land$ \\
            $$\bigwedge_{\langle e, n \rangle \in \Call{succ}{T}}(\Call{$\delta$}{X_T, U_e, C_T, X_n} \land U_e = \Call{action}{e} \land \Call{treeFormula}{k, n})$$
        }
        \EndIf
        \EndFunction
        \algstore{tf1}
    \end{algorithmic}

    \begin{algorithmic}[1]
        \algrestore{tf1}
        \Function{\textoverline{treeFormula}}{$k, T$}
        \If{$\Call{height}{k, T} = 0$}
        \State \Return{\Call{E}{$X_{T}$}}
        \Else
        \State \Return{ $\Call{E}{X_{T}} \lor$ \\
        $$\bigvee_{\langle e, n \rangle \in \Call{succ}{T}}(\Call{$\delta$}{X_T, U_T, C_e, X_n} \land C_e = \Call{action}{e} \land \Call{\textoverline{treeFormula}}{k, n})$$ }
        \EndIf
        \EndFunction
    \end{algorithmic}
\end{algorithm}

\section{Optimisations}

In this section we present several optimisations to the algorithm.

\subsection{Bad State Learning}

The most important optimisation that allows the algorithm to avoid much of the search space is to record states that are known to be losing for one player. On subsequent calls to the SAT solver we encode these states in the candidate strategy formula (see Algorithm~\ref{alg:treeFormulaLearning}). Thus the algorithm avoids choosing moves that lead to states that are already known to be losing.

Bad states are learned from failed attempts to find a candidate. If the SAT solver cannot find a candidate strategy for a given abstract game tree that means that there is a fixed prefix in the game tree for which the current player can never win. The state reached by playing the moves in the prefix must then be a losing state with some caveats. If the state is at the node with height $k$ and losing for the environment then we know that the environment cannot force to the error set in $k$ rounds. We do not know if the environment can force to the error set in $> k$ rounds. Therefore we record losing states for the environment in an array of sets of states $B^e$ indexed by the height at which the set is losing. For the controller, a losing state is losing for any run of length $>= k$. In practical use we are uninterested in controller strategies that make use of states that would lose should the game be extended to a longer bound so we merely maintain a single set of controller losing states $B^c$.

Additional states can be learned by expanding a single state into a set of losing states by greedily testing each variable of the state for inclusion in a \emph{cube} of states. This technique is well known in the literature and can be efficiently implemented using a SAT solver capable of solving under assumptions~\cite{Een03}. It is shown in Algorithm~\ref{alg:boundedLearning}.

\begin{algorithm}
    \caption{Modified Tree Formulas with Bad State Avoidance}
    \label{alg:treeFormulaLearning}
    \begin{algorithmic}[1]
        \Function{treeFormula}{$k, T$}
        \If{$\Call{height}{k, T} = 0$}
        \State \Return{ $\lnot \Call{$B^c$}{X_{T}}$ }
        \Else
        \State \Return{$\lnot \Call{$B^c$}{X_{T}} \land$ \\
            $$\bigwedge_{\langle e, n \rangle \in \Call{succ}{T}}(\Call{$\delta$}{X_T, U_e, C_T, X_n} \land U_e = \Call{action}{e} \land \Call{treeFormula}{k, n})$$
        }
        \EndIf
        \EndFunction
        \algstore{tf1}
    \end{algorithmic}

    \begin{algorithmic}[1]
        \algrestore{tf1}
        \Function{\textoverline{treeFormula}}{$k, T$}
        \If{$\Call{height}{k, T} = 0$}
        \State \Return{\Call{E}{$X_{T}$}}
        \Else
        \State \Return{\Call{$B^e$[\Call{height}{k,t}]}{$X_{T}$} $\lor$ \\
        $$\bigvee_{\langle e, n \rangle \in \Call{succ}{T}}(\Call{$\delta$}{X_T, U_T, C_e, X_n} \land C_e = \Call{action}{e} \land \Call{\textoverline{treeFormula}}{k, n})$$ }
        \EndIf
        \EndFunction
    \end{algorithmic}
\end{algorithm}

\subsection{Strategy Shortening}

Learning new bad states means reducing the search space for the algorithm. It follows that it is better to learn states earlier in the algorithm's execution. One problem with relying on SAT calls that assume cooperation is that there is no urgency to the returned candidate strategies. Consider the running example: the environment can reach the error set by setting \texttt{request} to 2 during two rounds. However, in the empty abstract game tree of a bounded game of length 3 or longer, there is no reason for the SAT solver to make the first action one of the requesting rounds if it can assume the environment will never grant any resources. The first action is important because the candidate strategy is derived from that. The candidate is what the opponent has the chance to respond to, so if the candidate does not do anything useful the opponent's response has the freedom to be equally apathetic about reaching its goal. This leads to much of the search space being explored unnecessarily until we learn a losing state.

Encouraging the SAT solver to find \emph{shorter} strategies is a successful heuristic for mitigating this issue. Whilst it does require more SAT calls per call to \textsc{findCandidate} it can be efficiently implemented using incremental SAT solving and during our benchmarking we found the cost to be worthwhile. A strategy is shorter if following the strategy leads to a known bad state for the opponent is fewer game rounds. For the environment this is clearly analogous to reaching the error set sooner. For the controller it is less clear but intuitively states that are environment losing at a certain height are more likely to be \emph{safe} states from which the controller may be able to force a loop.

\begin{algorithm}
    \caption{Strategy Shortening}
    \label{alg:strategyShortening}
    \begin{algorithmic}[1]
        \Function{shorten}{$p, s, k, T$}
            \State $\hat{T} \gets $ \Call{extend}{$T$}
            \State $f \gets $ \IfElse{$p = \texttt{cont}$}{\Call{treeFormula}{$k, \hat{T}$}}{\Call{\textoverline{treeFormula}}{$k, \hat{T}$}} \EndIfElse
            \State $prev \gets \top$
            \For{$l \in leaves(gt)$}
                \State $n \gets $ \Call{root}{$l$}
                \While{\Call{height}{$k, n$} $\neq 0$}
                    \If{$p = \texttt{cont}$}
                        \State $a \gets $ \Call{$B^e$[\Call{height}{$k, l$}]}{$X_n$}
                    \Else
                        \State $a \gets $ \Call{$E$}{$X_n$}
                    \EndIf
                    \State $sol \gets $ \Call{SATWithAssumptions}{$prev \land a$, $s(X_{\hat{T}}) \land f$}
                    \If{$sol \neq \texttt{NULL}$}
                        \State $prev \gets prev \land a$
                        \Break
                    \EndIf
                        
                    \State $n \gets $ \Call{succ}{$n$}
                \EndWhile
            \EndFor
            \State \Return $sol$
        \EndFunction
    \end{algorithmic}
\end{algorithm}

\subsection{Default Actions}

During the search for a candidate strategy the SAT solver selects actions for the opponent as though the players are cooperating. Sometimes the result is an action that will always fail for the opponent. In many specifications the environment is given the option to fail as a way of modelling errors. For example, in a network driver specification error transitions may be used to model failed connections. When such a transition exists it will often be selected by the SAT solver (especially when the strategy shortening optimisation is enabled). Constantly selecting a bad action for the opponent significantly affects the performance of the algorithm because no bad states can be learned and the solver must refine the game abstraction to avoid the bad action. Additionally, if a candidate strategy was found by relying on a bad action then it will usually need to be backtracked. 

To avoid problematic action selection the solver can instead use some heuristic to select the arbitrary action required in the SAT call in \textsc{findCandidate}. This does not affect the correctness of the algorithm. If no candidate can be found with the opponent playing an arbitrary action then clearly the selected action (or a different opponent action that is winning) would have eventually been refined into the abstract game if the opponent instead cooperated. A simple action selection heuristic has been observed to improve the performance of the solver during benchmarking. Before the main algorithm executes two SAT calls are made with formulas constructed from \textsc{treeFormula} and \textsc{\textoverline{treeFormula}} called on an empty abstract game tree. From the result a mapping of height to \emph{default action} is made for each player. During \textsc{findCandidate} calls the arbitrary opponent actions are taken from the corresponding map at the appropriate height.

\section{Correctness}

Completeness of the algorithm follows from the completeness of the backtracking search. In the worst case the algorithm will construct the entire concrete game tree and effectively expand all quantifiers. Soundness follows from the existential search of the SAT solver in \textsc{findCandidate}. The algorithm terminates after searching for a candidate strategy on an abstract game tree with actions fixed only for the opponent. If no candidate can be found with the opponent restricted in this way then no strategy exists for the player.

\section{Discussion}

The design of the algorithm is motivated by the desire to solve bounded safety games whilst avoiding the potential state explosion of computing the winning set. The key insight is to shift the emphasis from finding a winning set to finding winning strategies. The shift is made possible by searching for runs in an abstraction of the game and using the results to refine the abstraction. The advantage of this approach is that even when the winning set is difficult to represent symbolically (via a BDD or similar) a winning strategy may still be found. The reverse is also true: if the winning strategy requires too much branching it will become intractable to construct it using this algorithm. The difference can be likened to breadth-first versus depth-first search, the appropriate algorithm depends on the particular problem instance to be solved.

\section{Summary}

In this chapter I presented the fundamental building block of this thesis, a new algorithm for solving bounded realisability. In later chapters I will explain extensions to this algorithm to increase its applicability and in Chapter~\ref{ch:evaluation} I will present results and an evaluation of the contribution of this.

\begin{itemize}
    \item Here we introduce an algorithm for solving synthesis games that are bounded to a fixed number of game rounds. The algorithm is a counterexample guided abstraction refinement framework in which abstractions of the game are constructed from candidate strategies for the players. This is done in a way that allows a candidate strategy to be checked for a spoiling strategy by playing the game abstraction on behalf of the opponent. Spoiling strategies are counterexamples to a strategy that may be used for refinement.

    \item The design of the algorithm is inspired by the exponential blow up that can result from constructing a symbolic representation of the winning region as a BDD. In this algorithm the winning region is never computed although some winning states are learned as an optimisation to prune the search tree. In Chapter~\ref{ch:unbounded} we will see how this algorithm may be extended to unbounded synthesis by approximating the winning region during the execution of the algorithm.

\end{itemize}
