\chapter{Evaluation}
\label{ch:evaluation}

\section{Unbounded realisability}

We evaluate our approach on the benchmarks of the 2015 synthesis competition
(SYNTCOMP'15). Each benchmark comprises of controllable and uncontrollable
inputs to a circuit that assigns values to latches. One latch is configured as
the error bit that determines the winner of the safety game. The benchmark
suite is a collection of both real-world and toy specifications including
generalised buffers, AMBA bus controllers, device drivers, and converted LTL
formulas.  Descriptions of many of the benchmark families used can be found in
the 2014 competition report~\cite{jacobs2015}. 

The implementation of our algorithm uses \textsc{Glucose}~\cite{audemard2014}
for SAT solving and \textsc{Periplo}~\cite{rollini2013} for interpolant
generation. We intend to open source the tool for SYNTCOMP'16. The benchmarks
were run on a cluster of Intel Quad Core Xeon E5405 2GHz CPUs with 16GB of
memory.  The solvers were allowed exclusive access to a node for one hour to
solve an instance. 

The results of our benchmarking are shown, along with the synthesis competition
results~\cite{syntcompedacc}, in Table~\ref{tab:syntcomp}. The competition was
run on Intel Quad Core 3.2Ghz CPUs with 32GB of memory, also on isolated nodes
for one hour per instance. The competition results differ significantly from
our own benchmarks due to this more powerful hardware.  For our benchmarks we
report only the results for solvers we were able to run on our cluster. The
unique column lists the number of instances that only that tool could solve in
the competition (excluding our solver). In brackets is the number of instances
that only that tool could solve, including our solver.

\begin{table}
    \centering
    \setlength{\tabcolsep}{8pt}
    \begin{tabular}{l | r | r | r }
        \textbf{Solver} & \textbf{Solved} & \textbf{Solved} & \textbf{Unique} \\
                        & \textbf{(Competition)} & \textbf{(Benchmarks)} & \\
        \hline
        Simple BDD Solver (2) & 195 & 189 & 10 \textit{(6)} \\
        AbsSynthe (seq2) & 187 & 139 & 2 \\
        Simple BDD Solver (1) & 185 & 175 & \\
        AbsSynthe (seq3) & 179 & 134 & \\
        Realizer (sequential) & 179 & & \\
        AbsSynthe (seq1) & 173 & 139 & 1 \\
        Demiurge (D1real) & 139 & 136 & 5 \textit{(2)} \\
        Aisy & 98 & \\
        \textit{Unbounded Synthesis} & & \textit{103} & \textit{12} \\
    \end{tabular}
    \caption{Synthesis Competition 2015 Results}
    \label{tab:syntcomp}
\end{table}

Our implementation was able to solve 103 out of the 250 specification in the
alloted time, including 12 instances that were not solved by any other solver
in the sequential track of the competition. The unique instances we solved are
listed in Table~\ref{tab:unique}.

\begin{table}[h]
    \centering
    \setlength{\tabcolsep}{16pt}
    \begin{tabular}{l l}
        1. \texttt{6s216rb0\_c0to31} & 7. \texttt{driver\_c10n} \\
        2. \texttt{cnt30y} &  8. \texttt{stay18y} \\
        3. \texttt{driver\_a10n} & 9. \texttt{stay20n} \\
        4. \texttt{driver\_a8n} & 10. \texttt{stay20y} \\
        5. \texttt{driver\_b10y} & 11. \texttt{stay22n} \\
        6. \texttt{driver\_b8y} & 12. \texttt{stay22y} \\
    \end{tabular}
    \caption{Instances uniquely solved by our approach}
    \label{tab:unique}
\end{table}

Five of the instances unique to our solver are device driver instances and
another five are from the \texttt{stay} family. This supports the hypothesis
that different game solving methodologies perform better on certain classes of
specifications.

We also present a cactus plot of the number of instances solved over time
(Figure~\ref{fig:cactus}). We have plotted the best configuration of each
solver we benchmarked. The solvers shown are
\textsc{Demiurge}~\cite{bloem2014}, the only SAT-based tool in the competition,
the winner of the sequential realisability track \textsc{Simple BDD Solver
2}~\cite{walker2014}, and AbsSynthe (seq3) \cite{brenguier2014}.

\begin{figure}[h]
    \centering
    \begin{tikzpicture}[y=0.03cm, x=0.15cm]
        %axis
        \draw (0,0) -- coordinate (x axis mid) (60,0);
            \draw (0,0) -- coordinate (y axis mid) (0,200);
            %ticks
            \foreach \x in {0,10,...,60}
                \draw (\x,1pt) -- (\x,-3pt)
                node[anchor=north] {\x};
            \foreach \y in {0,25,...,200}
                \draw (1pt,\y) -- (-3pt,\y) 
                    node[anchor=east] {\y}; 
        %labels      
        \node[below=0.8cm] at (x axis mid) {Time (minutes)};
        \node[rotate=90, above left=1cm and 0.7cm] at (y axis mid) {Instances solved};
        
        %Plots
        \draw plot[mark=o, mark size=1pt, mark options={fill=white,color=red}] file {bench_termite.dat};
        \draw plot[mark=+, mark options={fill=white,color=blue}] file {bench_demiurge.dat};
        \draw plot[mark=triangle, mark options={fill=white}] file {bench_simple2.dat};
        \draw plot[mark=x, mark options={fill=white,color=green}] file {bench_abs1.dat};
%%%        \draw plot[mark=x, mark options={fill=white,color=green}] file {bench_abs2.dat};
%%%        \draw plot[mark=x, mark options={fill=white,color=green}] file {bench_abs3.dat};

        \begin{scope}[shift={(30,20)}] 
            \draw (0,0) -- 
                plot[mark=o, mark size=1pt, mark options={fill=white,color=red}] (0.25,0) -- (0.5,0) 
                node[right]{Unbounded synthesis};
            \draw[yshift=\baselineskip] (0,0) -- 
                plot[mark=x, mark options={fill=white,color=green}] (0.25,0) -- (0.5,0)
                node[right]{AbsSynthe (seq2)};
            \draw[yshift=2\baselineskip] (0,0) -- 
                plot[mark=+, mark options={fill=white,color=blue}] (0.25,0) -- (0.5,0)
                node[right]{Demiurge};
            \draw[yshift=3\baselineskip] (0,0) -- 
                plot[mark=triangle, mark options={fill=white}] (0.25,0) -- (0.5,0)
                node[right]{Simple BDD Solver 2};
        \end{scope}

    \end{tikzpicture}
    \caption{Number of instances solved over time.}
    \label{fig:cactus}
\end{figure}

Whilst our solver is unable to solve as many instances as other tools, it was
able to solve more unique instances than any solver in the competition. This
confirms that our methodology is able to fill gaps in a state of the art
synthesis toolbox by more efficiently solving instances that are hard for other
techniques. For this reason our solver would be a worthwhile addition to a
portfolio solver. In the parallel track of the competition, \textsc{Demiurge}
uses a suite of 3 separate but communicating solvers. The solvers
relay unsafe states to one another, which is compatible with the set $B^M$ in
our solver. This technique can already solve each of the unique instances
solved by our solver but there may still be value in the addition of this work
to the portfolio.  It remains future work to explore this possibility.
