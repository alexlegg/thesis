\chapter{Conclusion}
\label{ch:conclusion}

Controller synthesis may one day become the tool of choice for designers of reactive systems. As an approach to software correctness its advantages are clear but the future of synthesis is held back by computational complexity. In this thesis I presented a technique that takes one small step towards feasibility by focussing on a particular class of specifications.

The basis of this methodology is a counterexample guided bounded realisability algorithm. The algorithm constructs abstractions of the safety game and existentially searches for player strategies. The approach is similar to existing QBF methodologies but is able to perform better than those by taking advantage of knowledge of the structure of the problem. Bounded realisability is implemented in two tools: \eva and \termitesat. The latter of these is open source and available online. The approach was evaluated by benchmarking \eva against two QBF solvers: \textsc{depQBF} and \textsc{RAReQS}, as well as a BDD-based driver synthesis tool \textsc{Termite}. The results showed that bounded synthesis was able to outperform the BDD methods on classes of specifications for which the BDD representation of the winning region is exponential in the number of state variables.

I also presented an algorithm to extract the strategy from the certificate generated during bounded realisability. A certificate tree contains all of the actions required to prove that the game is realisable but does not contain enough information to construct a player strategy. I proposed an approach that uses Craig interpolation construct the set of states from which a player should choose a particular action in the certificate tree.  Strategy extraction was implemented as part of \eva and benchmarking showed that it introduces a linear overhead.

Additionally I presented an approach that extends the algorithm to solve unbounded games. The bounded realisability algorithm is able to show that there is no counterexamples traces of a certain length. In order to show that there is no counterexample of any length I proposed an approach that overapproximates the set of winning states for the controller. Overapproximation is achieved by carefully constructing interpolants in such a way that maintains two invariants that ensure that a fixed point in the computation is sufficient to prove realisability.  Similar to the bounded realisability algorithm, this unbounded realisability approach can perform well in cases where a compact BDD representation of the winning region does not exist. 

I implemented unbounded realisability in the open source \termitesat tool and submitted it to the synthesis competition. Although the submission was not competitive by itself it was able to solve several problem instances that no other solver could. I also implemented a hybrid approach within the tool that runs unbounded realisability in parallel with a BDD solver and shares some learned states. This hybrid approach performed very well in the parallel track of the competition. These results support the argument that unbounded synthesis is useful as part of a complete synthesis toolkit in order to solve problem instances that are infeasible for BDD based approaches.

In summary, I have presented a synthesis algorithm inspired by counterexample guided QBF and bounded model checking. I implemented the algorithm and demonstrated its usefulness in cases where traditional approaches are infeasible.
