\chapter{Evaluation}
\label{ch:evaluation}

\newcommand{\eva}[0]{\textsc{EvaSolver}\xspace}
\newcommand{\termitesat}[0]{\textsc{TermiteSAT}\xspace}
\newcommand{\ignore}[1]{}

\definecolor{RYB1}{RGB}{238, 46, 47}
\definecolor{RYB2}{RGB}{0, 140, 72}
\definecolor{RYB3}{RGB}{24, 90, 169}
\definecolor{RYB4}{RGB}{244, 125, 35}
\definecolor{RYB5}{RGB}{102, 44, 145}
\definecolor{RYB6}{RGB}{180, 56, 148}

\pgfplotscreateplotcyclelist{plotcolours}{%
{RYB1!80!black, mark options={fill=RYB1,color=RYB1}, mark=pentagon*},
{RYB3!80!black, mark options={fill=RYB3,color=RYB3}, mark=triangle*},
{RYB2!80!black, mark options={fill=RYB2,color=RYB2}, mark=halfsquare*},
{RYB4!80!black, mark options={fill=RYB4,color=RYB4}, mark=diamond*},
{RYB5!80!black, mark options={fill=RYB5,color=RYB5}, mark=halfcircle*},
{RYB6!80!black, mark options={fill=RYB6,color=RYB6}, mark=x}}

\pgfplotsset{
    every axis plot/.append style={line width=0.8pt},
    every axis/.append style={
        axis lines=middle, 
        x label style={at={(axis description cs:0.5,-0.1)},anchor=north},
        y label style={at={(axis description cs:-0.1,.5)},rotate=90,anchor=south},
        legend style={draw=none,cells={anchor=west},anchor=north west,at={(axis description cs:0.05,1)}},
        cycle list name=plotcolours}
}

In this chapter I present benchmarking results for the implementations of each previous chapter. Bounded realisability is implemented in a tool called \eva as joint work between Nina Narodytska and myself. \eva was built in C++ and is based on the source code of \textsc{RAReQS}~\cite{Janota12}. The tool calls out to Glucose~\cite{Audemard09} for SAT solving. Strategy extraction was later added to \eva using PeRIPLO~\cite{Rollini13} to construct interpolants. The implementation also uses cudd~\cite{Somenzi01} to reduce interpolants into cubes via BDDs. I implemented unbounded realisability in a separate open source tool, \termitesat~\cite{TermiteSAT}, which contains a reimplementation of the bounded realisability algorithm. \termitesat is written in Haskell and also uses Glucose for SAT solving, PeRIPLO for interpolant construction, and cudd to reduce interpolants to cubes. \termitesat was submitted to the 2016 synthesis competition and the results are shown below. As part of that submission I added a hybrid mode to \termitesat that runs the unbounded synthesis algorithm in parallel with Simple BDD Solver~\cite{Walker14b}.

\section{Bounded realisability}

The algorithm is evaluated on four families of benchmarks derived from driver synthesis problems. Bounded realisability is only able to prove the absence of counterexample traces up to a certain length for safety games so the tool is instead evaluated on benchmarks formulated as the dual reachability game. \eva solves games in which the controller must \emph{reach} a goal state. The benchmarks are equivalent to unrealisable safety specifications.  Each benchmark models the data path of one of four I/O devices in the abstracted form.  In particular, we model the transmit buffer of an Ethernet adapter, the send queue of a UART serial controller, the command queue of an SPI Flash controller, and the IDE hard disk DMA descriptor list.   Models are parameterised by the size of the corresponding data structure.  Specifications are written in a simple input language based on the NuSMV syntax~\cite{Henzinger03}.   The transition relation of the game is given in the form of variable update functions $s' := f(\mathcal{S}, \mathcal{U}, \mathcal{C})$, one for each state variable $s \in \mathcal{S}$.

We compare our solver against two existing approaches to solving games.  First, we encode input specifications as QBF instances and solve them using two state-of-the-art QBF solvers: \textsc{RAReQS}~\cite{Janota12} and \textsc{depQBF}~\cite{Lonsing10}, having first run them through the bloqqer~\cite{Biere11} preprocessor.  Second, we solve our benchmarks using the Termite~\cite{Walker14} BDD-based solver that uses dynamic variable reordering, variable grouping, transition relation partitioning, and other optimisations. 

Our experiments, summarised in \Cref{fig:uart,fig:ide,fig:spi,fig:ethernet}, show that off-the-shelf QBF solvers are not well-suited for solving games.  Although our algorithm is inspired by \textsc{RAReQS}, we achieve much better performance, since our solver takes into account the structure of the game, rather than treating it as a generic QBF problem.  

\begin{figure}[ht]
    \centering
    \pgfplotsset{width=\textwidth, height=0.5\textheight}
    \begin{tikzpicture}
        \begin{axis}[xmin=0, xmax=50, ymin=0, ymax=4000, xlabel={Instance size ($| \cS |$)}, ylabel={Time (seconds)}]
            \addplot file {data/bench_uart_eva.dat};
            \addplot file {data/bench_uart_termite.dat};
            \addplot file {data/bench_uart_rareqs.dat};
            \addplot file {data/bench_uart_depqbf.dat};
            \legend{EvaSolver,Termite,RAReQS,depQBF}
        \end{axis}
    \end{tikzpicture}
    \caption{UART}
    \label{fig:uart}
\end{figure}

\begin{figure}[hb]
    \centering
    \pgfplotsset{width=\textwidth, height=0.375\textheight}
    \begin{tikzpicture}
        \begin{axis}[xmin=15, xmax=50, ymin=0, ymax=3000, xlabel={Instance size ($| \cS |$)}, ylabel={Time (seconds)}]
            \addplot file {data/bench_ide_eva.dat};
            \addplot file {data/bench_ide_termite.dat};
%%%            \addplot file {data/bench_ide_rareqs.dat};
%%%            \addplot file {data/bench_ide_depqbf.dat};
            \legend{EvaSolver,Termite,RAReQS,depQBF}
        \end{axis}
    \end{tikzpicture}
    \caption{IDE DMA}
    \label{fig:ide}
\end{figure}

\begin{figure}[ht]
    \centering
    \pgfplotsset{width=\textwidth, height=0.375\textheight}
    \begin{tikzpicture}
        \begin{axis}[xmin=0, xmax=100, ymin=0, ymax=3000, xlabel={Instance size ($| \cS |$)}, ylabel={Time (seconds)}]
            \addplot file {data/bench_spi_eva.dat};
            \addplot file {data/bench_spi_termite.dat};
            \addplot file {data/bench_spi_rareqs.dat};
            \addplot file {data/bench_spi_depqbf.dat};
            \legend{EvaSolver,Termite,RAReQS,depQBF}
        \end{axis}
    \end{tikzpicture}
    \caption{SPI}
    \label{fig:spi}
\end{figure}

\begin{figure}[hb]
    \centering
    \pgfplotsset{width=\textwidth, height=0.5\textheight}
    \begin{tikzpicture}
        \begin{axis}[xmin=0, xmax=75, ymin=0, ymax=4000, xlabel={Instance size ($| \cS |$)}, ylabel={Time (seconds)}]
            \addplot file {data/bench_queue_eva.dat};
            \addplot file {data/bench_queue_termite.dat};
            \addplot file {data/bench_queue_rareqs.dat};
            \addplot file {data/bench_queue_depqbf.dat};
            \legend{EvaSolver,Termite,RAReQS,depQBF}
        \end{axis}
    \end{tikzpicture}
    \caption{Ethernet}
    \label{fig:ethernet}
\end{figure}


All four benchmarks have very large sets of winning states.  
Nevertheless, in the UART and IDE benchmarks, Termite is able to 
represent winning states compactly with only a few thousand BDD 
nodes.  It scales well and outperforms \eva on these benchmarks.  
However, in the two other benchmarks, Termite does not find a 
compact BDD-based representation of the winning set.  \eva 
outperforms Termite on these benchmarks as it does not try to 
enumerate all winning states.

{Detailed performance analysis shows that 
abstract game trees generated in our benchmarks had average 
branching factors in the range between $1.03$ and $1.2$, with 
the maximal depth of the trees ranging from $3$ to $58$.  This
confirms the the key premise behind the design of \eva, namely, 
solving real-world synthesis problems requires considering only a 
small number of opponent moves in every state of the game.}  

\clearpage

\section{Strategy extraction}

%%%\eva implements an important optimisation whereby it generates multiple certificate trees for fragments of the original game, which enables computational learning of winning states.  Our strategy generation algorithm is invoked in an online fashion, whenever \eva computes a certificate tree for a fragment.  This results in multiple partial strategies, where each strategy is computed as described in the previous section.  We introduce an additional final step to \eva, which combines partial strategies into a global winning strategy for the original game.

Strategy extraction is evaluated on the same set of driver synthesis benchmarks.  \Cref{fig:uart2,fig:ide2,fig:spi2,fig:ethernet2} sumarise the results.  For each family, we show strategy generation time as a function of the number of state variables in the specification for a selection of the hardest instances of the family solved by \eva.  Specifically, we show the time it took the base realisability solver to determine the winner (the realisability line), as well as the total time taken to solve the game and compute the winning strategy (the total line).


Table~\ref{table:strategyextractionresults} shows a detailed breakdown of experimental results, including the number of state variables for each instance (\textbf{Vars}) and the total time taken by the solver (\textbf{Total(s)}), split between the time used to determine the winner (and generate certificate trees) (\textbf{Base(s)}) and the strategy generation time (\textbf{Strategy(s)}). The \textbf{OH} column shows the overhead of strategy extraction.

Profiling showed that non-negligible overhead was introduced by transferring CNFs from $\eva$'s internal representation to the representation used by the interpolation library.  This overhead can be almost completely eliminated with additional engineering effort.  Hence, I report the effective overhead (\textbf{EffOH}) of strategy extraction if this engineering effort had been done.

The \textbf{Size} column shows the size of the strategy, i.e., the number of $(W,a,k)$ tuples returned by the \textsc{GenLocalStrats} function.  The last three columns report on the use of the PeRIPLO interpolation library in terms of the number of interpolation operations performed by the algorithm when solving the instance, the average and the maximal size of interpolants returned by PeRIPLO.  The size of an interpolant is reported by the size of its corresponding BDD. This conversion is done by the tool as an optimisation to simplify interpolants into cubes.

These results show that strategy generation adds a modest overhead to the base solver.  Effective overheads are about $12\%$ for IDE and SPI, about $35\%$ for Ethernet and about $30\%$ for UART.  Most of this overhead is due to interpolant computation.  Moreover, experiments show that our algorithm scales linearly with the time taken by the base solver to determine the winner.  

\begin{figure}[ht]
    \centering
    \pgfplotsset{width=\textwidth, height=0.5\textheight}
    \begin{tikzpicture}
        \begin{axis}[xmin=10, xmax=40, ymin=0, ymax=4000, xlabel={Instance size ($| \cS |$)}, ylabel={Time (seconds)}]
            \addplot file {data/bench_strat_uart_total.dat};
            \addplot file {data/bench_strat_uart_real.dat};
            \legend{Total,Realisability}
        \end{axis}
    \end{tikzpicture}
    \caption{Strategy Extraction (UART)}
    \label{fig:uart2}
\end{figure}

\begin{figure}[hb]
    \centering
    \pgfplotsset{width=\textwidth, height=0.25\textheight}
    \begin{tikzpicture}
        \begin{axis}[xmin=25, xmax=35, ymin=0, ymax=2000, xlabel={Instance size ($| \cS |$)}, ylabel={Time (seconds)}]
            \addplot file {data/bench_strat_ide_total.dat};
            \addplot file {data/bench_strat_ide_real.dat};
            \legend{Total,Realisability}
        \end{axis}
    \end{tikzpicture}
    \caption{Strategy Extraction (IDE DMA)}
    \label{fig:ide2}
\end{figure}

\begin{figure}[ht]
    \centering
    \pgfplotsset{width=\textwidth, height=0.4375\textheight}
    \begin{tikzpicture}
        \begin{axis}[xmin=0, xmax=140, ymin=0, ymax=3500, xlabel={Instance size ($| \cS |$)}, ylabel={Time (seconds)}]
            \addplot file {data/bench_strat_spi_total.dat};
            \addplot file {data/bench_strat_spi_real.dat};
            \legend{Total,Realisability}
        \end{axis}
    \end{tikzpicture}
    \caption{Strategy Extraction (SPI)}
    \label{fig:spi2}
\end{figure}

\begin{figure}[hb]
    \centering
    \pgfplotsset{width=\textwidth, height=0.25\textheight}
    \begin{tikzpicture}
        \begin{axis}[ xmin=0, xmax=60, ymin=0, ymax=2000, xlabel={Instance size ($| \cS |$)}, ylabel={Time (seconds)}]
            \addplot file {data/bench_strat_ethernet_total.dat};
            \addplot file {data/bench_strat_ethernet_real.dat};
            \legend{Total,Realisability}
        \end{axis}
    \end{tikzpicture}
    \caption{Strategy Extraction (Ethernet)}
    \label{fig:ethernet2}
\end{figure}

\begin{table}[t]
\resizebox{\columnwidth}{!}{%
\begin{tabular}{r r r r r r r r r r}
\textbf{Vars} & \textbf{Total(s)} & \textbf{Base(s)} & \textbf{Strategy(s)} & \textbf{OH} & \textbf{EffOH} & \textbf{Size} & \textbf{INum} & \textbf{IAvg} & \textbf{IMax}  \\

\hline
\multicolumn{10}{c}{IDE benchmark} \\
\hline
26 & 32.62 & 25.42 & 7.20 & 1.28 & 1.16 & 50    & 48 & \ignore{1193} 23     & \ignore{24155} 118   \\
28 & 42.20 & 35.49 & 6.72 & 1.19 & 1.10 & 59    & 52 & \ignore{2489} 27     & \ignore{39384} 119   \\
30 & 60.04 & 51.93 & 8.11 & 1.16 & 1.08 & 92    & 43 & \ignore{72} 17       & \ignore{1583} 148    \\
32 & 115.11 & 107.35 & 7.77 & 1.07 & 1.04 & 60  & 36 & \ignore{18} 14     & \ignore{85} 27       \\
34 & 283.08 & 227.67 & 55.40 & 1.24 & 1.21 & 159& 49 & \ignore{2150} 15 & \ignore{103941} 38   \\
\hline
\multicolumn{10}{c}{SPI benchmark} \\
\hline
15 & 0.35 & 0.26 & 0.09 & 1.36 & 1.26 & 8             & 5   & \ignore{11} 9 & \ignore{21} 9   \\
22 & 0.94 & 0.72 & 0.22 & 1.31 & 1.19 & 15            & 12  & \ignore{10} 10 & \ignore{22} 18  \\
29 & 2.46 & 1.90 & 0.56 & 1.29 & 1.16 & 22            & 17  & \ignore{12} 10 & \ignore{25} 18  \\
36 & 3.56 & 2.91 & 0.65 & 1.22 & 1.11 & 107           & 22  & \ignore{11} 10 & \ignore{22} 18  \\
43 & 9.11 & 7.09 & 2.03 & 1.29 & 1.13 & 166           & 27  & \ignore{11} 10 & \ignore{21} 14  \\
50 & 16.20 & 12.85 & 3.35 & 1.26 & 1.12 & 233         & 32  & \ignore{11} 11 & \ignore{23} 18  \\
57 & 25.00 & 19.86 & 5.14 & 1.26 & 1.12 & 322         & 37  & \ignore{12} 11 & \ignore{21} 18  \\
64 & 38.48 & 31.48 & 7.00 & 1.22 & 1.10 & 416         & 42  & \ignore{12} 11 & \ignore{24} 18  \\
71 & 57.88 & 47.94 & 9.94 & 1.21 & 1.09 & 70          & 47  & \ignore{12} 12 & \ignore{21} 18  \\
78 & 91.51 & 75.02 & 16.49 & 1.22 & 1.10 & 636        & 52  & \ignore{12} 12 & \ignore{22} 19  \\
85 & 141.10 & 116.71 & 24.39 & 1.21 & 1.09 & 773      & 57  & \ignore{13} 12 & \ignore{23} 20  \\
92 & 193.96 & 162.05 & 31.91 & 1.20 & 1.09 & 917      & 62  & \ignore{13} 13 & \ignore{24} 21  \\
99 & 309.44 & 256.88 & 52.55 & 1.20 & 1.09 & 1059     & 67  & \ignore{13} 13 & \ignore{22} 22  \\
106 & 449.49 & 377.48 & 72.00 & 1.19 & 1.09 & 1223    & 72  & \ignore{13} 13 & \ignore{23} 23  \\
113 & 1645.44 & 1543.84 & 101.60 & 1.07 & 1.03 & 117  & 77  & \ignore{14} 13 & \ignore{24} 24  \\
120 & 901.95 & 830.17 & 71.77 & 1.09 & 1.04 & 1637    & 82  & \ignore{14} 14 & \ignore{25} 25  \\
127 & 2259.65 & 2143.40 & 116.25 & 1.05 & 1.02 & 139  & 87  & \ignore{14} 14 & \ignore{26} 26  \\
134 & 2385.74 & 2193.65 & 192.09 & 1.09 & 1.04 & 152  & 92  & \ignore{14} 14 & \ignore{27} 27  \\
\hline
\multicolumn{10}{c}{Ethernet benchmarks} \\
\hline
14 & 0.06 & 0.03 & 0.02 & 1.60 & 1.52 & 2             & 1    & \ignore{24} 13  & \ignore{24} 13    \\
17 & 0.49 & 0.29 & 0.20 & 1.70 & 1.45 & 21            & 7    & \ignore{33} 16  & \ignore{87} 30    \\
20 & 1.97 & 1.14 & 0.82 & 1.72 & 1.45 & 176           & 15   & \ignore{41} 16  & \ignore{110} 26   \\
23 & 5.39 & 3.23 & 2.16 & 1.67 & 1.39 & 185           & 25   & \ignore{64} 23  & \ignore{180} 42   \\
26 & 14.61 & 7.94 & 6.66 & 1.84 & 1.48 & 266          & 36   & \ignore{100} 24 & \ignore{347} 45   \\
29 & 27.41 & 15.71 & 11.70 & 1.74 & 1.43 & 677        & 44   & \ignore{155} 24 & \ignore{779} 48   \\
32 & 58.02 & 35.38 & 22.64 & 1.64 & 1.36 & 208        & 61   & \ignore{136} 28 & \ignore{676} 55   \\
35 & 111.69 & 69.26 & 42.43 & 1.61 & 1.35 & 351       & 80   & \ignore{141} 31 & \ignore{933} 75   \\
38 & 238.09 & 151.21 & 86.89 & 1.57 & 1.33 & 545      & 116  & \ignore{184} 32 & \ignore{1081} 63  \\
41 & 513.61 & 321.78 & 191.82 & 1.60 & 1.34 & 1525    & 154  & \ignore{184} 35 & \ignore{1123} 72  \\
44 & 845.51 & 530.68 & 314.83 & 1.59 & 1.34 & 2159    & 191  & \ignore{276} 37 & \ignore{2253} 64  \\
47 & 903.79 & 590.19 & 313.60 & 1.53 & 1.30 & 1547    & 228  & \ignore{261} 38 & \ignore{1780} 71  \\
50 & 1368.23 & 875.90 & 492.33 & 1.56 & 1.33 & 1670   & 236  & \ignore{372} 38 & \ignore{2292} 85  \\
\hline
\multicolumn{10}{c}{UART benchmarks} \\
\hline
15 & 2.86 & 2.19 & 0.67 & 1.31 & 1.19 & 12             & 40  & \ignore{28} 6   & \ignore{90} 14    \\
20 & 3.16 & 2.33 & 0.83 & 1.36 & 1.20 & 20             & 14  & \ignore{35} 12  & \ignore{155} 23   \\
21 & 10.06 & 6.96 & 3.09 & 1.44 & 1.24 & 35            & 34  & \ignore{48} 9   & \ignore{306} 26   \\
26 & 27.89 & 18.55 & 9.34 & 1.50 & 1.27 & 65           & 60  & \ignore{92} 13  & \ignore{730} 41   \\
27 & 63.68 & 41.49 & 22.20 & 1.53 & 1.29 & 93          & 94  & \ignore{96} 13  & \ignore{825} 44   \\
28 & 137.24 & 90.68 & 46.56 & 1.51 & 1.27 & 103        & 136 & \ignore{138} 13 & \ignore{1356} 42   \\
29 & 270.66 & 178.75 & 91.92 & 1.51 & 1.27 & 134       & 184 & \ignore{212} 15 & \ignore{2806} 47   \\
34 & 553.29 & 360.76 & 192.53 & 1.53 & 1.28 & 191      & 246 & \ignore{299} 16 & \ignore{6360} 54   \\
35 & 938.68 & 612.63 & 326.05 & 1.53 & 1.28 & 285      & 307 & \ignore{258} 16 & \ignore{7949} 69   \\
36 & 1525.99 & 995.25 & 530.74 & 1.53 & 1.28 & 410     & 382 & \ignore{348} 17 & \ignore{6408} 62   \\
37 & 2464.13 & 1611.45 & 852.68 & 1.53 & 1.28 & 950    & 456 & \ignore{414} 18 & \ignore{10592} 75  \\
38 & 3927.64 & 2577.39 & 1350.25 & 1.52 & 1.28 & 1223  & 546 & \ignore{504} 18 & \ignore{34431} 74  \\
39 & 6030.77 & 4031.98 & 1998.79 & 1.50 & 1.26 & 674   & 633 & \ignore{608} 18 & \ignore{29996} 72  \\
\hline
\end{tabular}}
\caption{Detailed strategy extraction results}
\label{table:strategyextractionresults}
\end{table}


%%%These results show that strategy extraction is efficient, scalable and robust.  The last property is particularly interesting, as existing strategy extraction algorithms for traditional game solvers, based on winning set compilation, have been reported to introduce significant variance across instances~\cite{Bloem_KS_14corr,Bloem_KS_13}.  A conclusive comparison can only be performed by evaluating both types of algorithms on a common set of benchmarks.  Such a comparison requires first extending $\eva$ to support unbounded safety and reachability games and is part of the future work.

\clearpage

\section{Unbounded realisability}

Unbounded realisability is evaluated on the benchmarks of the 2015 synthesis competition (SYNTCOMP'15). I also show the results of the 2016 competition, which \termitesat was submitted to. Each competition benchmark comprises of controllable and uncontrollable inputs to a circuit that assigns values to latches. One latch is configured as the error bit that determines the winner of the safety game. The benchmark suite is a collection of both real-world and toy specifications including generalised buffers, AMBA bus controllers, device drivers, and converted LTL formulas.  Descriptions of many of the benchmark families used can be found in the 2014 competition report~\cite{Jacobs15}. 

\subsection{Benchmarking}

The benchmarks were run on a cluster of Intel Quad Core Xeon E5405 2GHz CPUs with 16GB of memory.  The solvers were allowed exclusive access to a node for one hour to solve an instance.  The results of this benchmarking are shown, along with the synthesis competition results~\cite{syntcompedacc}, in Table~\ref{tab:syntcomp}. The competition was run on Intel Quad Core 3.2Ghz CPUs with 32GB of memory, also on isolated nodes for one hour per instance. The competition results differ significantly from our own benchmarks due to this more powerful hardware.  For our benchmarks we report only the results for solvers we were able to run on our cluster. The unique column lists the number of instances that only that tool could solve in the competition (excluding our solver). In brackets is the number of instances that only that tool could solve, including our solver.

\begin{table}[b]
    \centering
    \setlength{\tabcolsep}{8pt}
    \begin{tabular}{l | r | r | r }
        \textbf{Solver} & \textbf{Solved} & \textbf{Solved} & \textbf{Unique} \\
                        & \textbf{(Competition)} & \textbf{(Benchmarks)} & \\
        \hline
        Simple BDD Solver (2) & 195 & 189 & 10 \textit{(6)} \\
        AbsSynthe (seq2) & 187 & 139 & 2 \\
        Simple BDD Solver (1) & 185 & 175 & \\
        AbsSynthe (seq3) & 179 & 134 & \\
        Realizer (sequential) & 179 & & \\
        AbsSynthe (seq1) & 173 & 139 & 1 \\
        Demiurge (D1real) & 139 & 136 & 5 \textit{(2)} \\
        Aisy & 98 & \\
        \textit{Unbounded Synthesis} & & \textit{103} & \textit{12} \\
    \end{tabular}
    \caption{Synthesis Competition 2015 Results}
    \label{tab:syntcomp}
\end{table}

Our implementation was able to solve 103 out of the 250 specification in the alloted time, including 12 instances that were not solved by any other solver in the sequential track of the competition. The unique instances we solved are listed in Table~\ref{tab:unique}.

\begin{table}[h]
    \centering
    \setlength{\tabcolsep}{16pt}
    \begin{tabular}{l l}
        1. \texttt{6s216rb0\_c0to31} & 7. \texttt{driver\_c10n} \\
        2. \texttt{cnt30y} &  8. \texttt{stay18y} \\
        3. \texttt{driver\_a10n} & 9. \texttt{stay20n} \\
        4. \texttt{driver\_a8n} & 10. \texttt{stay20y} \\
        5. \texttt{driver\_b10y} & 11. \texttt{stay22n} \\
        6. \texttt{driver\_b8y} & 12. \texttt{stay22y} \\
    \end{tabular}
    \caption{Instances uniquely solved by our approach}
    \label{tab:unique}
\end{table}

Five of the instances unique to our solver are device driver instances and another five are from the \texttt{stay} family. This supports the hypothesis that different game solving methodologies perform better on certain classes of specifications.

I also present a cactus plot of the number of instances solved over time (Figure~\ref{fig:cactus}). We have plotted the best configuration of each solver we benchmarked. The solvers shown are \textsc{Demiurge}~\cite{Bloem14}, the only SAT-based tool in the competition, the winner of the sequential realisability track \textsc{Simple BDD Solver 2}~\cite{Walker14}, and AbsSynthe (seq3) \cite{Brenguier14}.

\begin{figure}
    \centering
    \pgfplotsset{width=\textwidth}
    \begin{tikzpicture}
        \begin{axis}[ xmin=0, xmax=60, ymin=0, ymax=200, xlabel={Time (minutes)}, ylabel={Instances solved},
                legend style={draw=none,cells={anchor=west},anchor=south east,at={(axis description cs:1,0.05)}}]
            \addplot file {data/bench_termite.dat};
            \addplot file {data/bench_demiurge.dat};
            \addplot file {data/bench_simple2.dat};
            \addplot file {data/bench_abs1.dat};
            \legend{TermiteSAT,Demiurge,Simple BDD Solver,AbsSynthe}
        \end{axis}
    \end{tikzpicture}
    \caption{Number of instances solved over time.}
    \label{fig:cactus}
\end{figure}

%%%Whilst our solver is unable to solve as many instances as other tools, it was
%%%able to solve more unique instances than any solver in the competition. This
%%%confirms that our methodology is able to fill gaps in a state of the art
%%%synthesis toolbox by more efficiently solving instances that are hard for other
%%%techniques. For this reason our solver would be a worthwhile addition to a
%%%portfolio solver. In the parallel track of the competition, \textsc{Demiurge}
%%%uses a suite of 3 separate but communicating solvers. The solvers
%%%relay unsafe states to one another, which is compatible with the set $B^M$ in
%%%our solver. This technique can already solve each of the unique instances
%%%solved by our solver but there may still be value in the addition of this work
%%%to the portfolio.  It remains future work to explore this possibility.

\subsection{Synthesis Competition Results}

I report the results of the sequential realisability track of the 2016 synthesis competition in Table~\ref{table:syntcomp16s}

\begin{table}
%%%    \begin{tabular}{r l r r}
%%%        1 & Simple BDD Solver (w/ Abstraction) & 175 & 1 \\
%%%        2 & Simple BDD Solver (w/ Abstraction 2) & 167 & 1 \\
%%%        3 & SafetySynth & 164 & 0 \\
%%%        Simple BDD Solver & 164 & 0 \\
%%%        5 & SafetySynth (Alt) & 163 & 0 \\
%%%        6 & AbsSynthe (S3) & 161 & 4 \\
%%%        7 & AbsSynthe (S2) & 151 & 0 \\
%%%        8 & SDF & 149 & 0 \\
%%%        9 & AbsSynthe (S1) & 147 & 0 \\
%%%        10 & Demiurge D1real & 129 & 6 \\
%%%        11 & TermiteSAT & 97 & 4 \\
%%%    \begin{end}
    \caption{SYNTCOMP'16: sequential realisability track}
    \label{table:syntcomp16s}
\end{table}

\begin{table}
    \centering
    \begin{enumerate}
        \item \texttt{stay18}
        \item \texttt{stay20n}
        \item \texttt{amba9match5}
        \item \texttt{6s191\_c0t8}
    \end{enumerate}
    \caption{SYNTCOMP'16: Instances uniquely solved by our approach}
    \label{tab:synt16unique}
\end{table}

\begin{figure}
    \centering
    \pgfplotsset{width=\textwidth}
    \begin{tikzpicture}
        \begin{axis}[ xmin=0, xmax=60, ymin=0, ymax=200, xlabel={Time (minutes)}, ylabel={Instances solved},
                legend style={draw=none,cells={anchor=west},anchor=south east,at={(axis description cs:1,0.05)}}]
            \addplot file {data/comp_sequential_TermiteSAT.dat};
            \addplot file {data/comp_sequential_SimpleBDDSolverwithAbstraction.dat};
            \addplot file {data/comp_sequential_DemiurgeD1real.dat};
            \addplot file {data/comp_sequential_AbsSyntheS3.dat};
            \addplot file {data/comp_sequential_SafetySynth.dat};
            \addplot file {data/comp_sequential_SDF.dat};
            \legend{TermiteSAT,Demiurge,Simple BDD Solver,AbsSynthe,SafetySynth,SDF}
        \end{axis}
    \end{tikzpicture}
    \caption{SYNTCOMP'16 sequential track: Instances solved over time}
    \label{fig:syntcomp16seq}
\end{figure}


\section{Summary}
