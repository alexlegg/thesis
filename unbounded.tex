\chapter{Unbounded Realisability}
\label{ch:unbounded}

\newtheorem*{exmpInt}{Example: Why we use interpolants}

In previous chapters I outlined an algorithm to solve bounded realisability games and an extension that can extract strategies from the result. Bounded realisability can be used to prove the existence of a winning strategy for the environment on the unbounded game by providing a witness. For the controller, the strongest claim that can be made is that the strategy is winning as long as the game does not extend beyond the maximum bound. The work described in this chapter can be used to address this by presenting another extension to the algorithm that solves unbounded realisability games.

The baseline solution to this problem is to set a maximum bound such that all runs in the unbounded game will be considered. The na\"ive approach is to use size of the state space as the bound ($2^\mathcal{S}$) so that all states may be explored by the algorithm. In model checking, a more nuanced approach is to use the diameter of the system~\cite{Biere99}, which is the smallest number $d$ such that for any state $x$ there is a path of length $\leq d$ to all other reachable states. An analogous approach for solving games would require computing the longest path the environment can enforce in the game and setting the bound to its length. Computing a sufficient bound is expensive and the subsequent bounded realisability check may be infeasible if the bound is high.

Instead I present an approach that iteratively solves games of increasing
bound while learning bad states from abstract games using Craig interpolation. We utilise the approximation properties of the interpolant to construct sets of states that underapproximate the total losing set for the controller. By underapproximating we avoid constructing a potentially large representation of this set that could be the cause of infeasibility in a BDD solver. Later in this chapter we will see that a careful construction of approximate sets enables a fixed point that is sufficient to prove the nonexistence of an environment-winning strategy.

\section{Algorithm}
\label{sec:unboundedAlgorithm}

%%%Recall the bounded realisability algorithm from Chapter~\ref{ch:bounded}. A safety game is solved with respect to a bound on the number of game rounds. The algorithm is a counterexample guided approach that constructs abstractions of the game in the form of game trees. An abstract game tree restricts one player to the actions that label edges in the tree. In Chapter~\ref{ch:strategy} I presented a strategy extraction procedure that utilises the abstract game tree as a certificate for the game. In this chapter I introduce a similar approach that extracts an approximation of winning states from the certificate tree.

Recall the bounded realisability algorithm from Chapter~\ref{ch:bounded}. In Section~\ref{sec:boundedLearning} I presented an optimisation that learns a subset of states losing for one of the players every time a candidate cannot be found for an abstract game. In this chapter I present I present an improved learning procedure that maintains certain properties for sets of learned states.  The bounded algorithm is reproduced in Algorithm~\ref{alg:unboundedBounded} with additional calls to the new learning procedure. The algorithm solves a game $(\mathcal{S}, \mathcal{U}, \mathcal{C}, \delta, s_0)$ with state variables $\mathcal{S}$, environment variables $\mathcal{U}$, controller variables $\mathcal{C}$, transition relation $\delta$, and initial states $s_0$. The safety condition of the game is defined by a set of error states $E$ that the controller must avoid and the environment must reach. 

\begin{algorithm}
    \begin{algorithmic}[1]
        \Function{solveAbstract}{$p, s, k, T$}
        \State $cand \gets $ \Call{findCandidate}{$p, s, k, T$} 
        \IIf{$k = 1$} \Return $cand$ \EndIIf 
        \State $T' \gets T$
        \Loop
            \IIf{$cand = \texttt{NULL}$} \Return $\texttt{NULL}$ \EndIIf 
            \State $\langle cex, l, u \rangle \gets $ \Call{verify}{$p, s, k, T, cand$} 
            \IIf{$cex = \False$} \Return $cand$ \EndIIf 
            \State $T' \gets $ \Call{append}{$T', l, u$} 
            \State $cand \gets $ \Call{solveAbstract}{$p, s, k, T'$} 
        \EndLoop
        \EndFunction
        \algstore{b1}
    \end{algorithmic}

    \begin{algorithmic}
        \algrestore{b1}
        \Function{findCandidate}{$p, s, k, T$}
        \State $\hat{T} \gets $ \Call{extend}{$T$}
            \State $f \gets $ \IfElse{$p = \texttt{cont}$}{\Call{treeFormula}{$k, \hat{T}$}}{\Call{\textoverline{treeFormula}}{$k, \hat{T}$}} \EndIfElse
            \State $sol \gets $ \Call{SAT}{$s(\mathcal{S}_{\hat{T}}) \land f$}
            \If{$sol = \texttt{unsat}$} 
                \LineComment{Unbounded solver learns states here}
                \State \IfElse{$p = \texttt{cont}$}{\Call{learn}{$s, \hat{T}$}}{\Call{\textoverline{learn}}{$s, \hat{T}$}}\label{alg:unboundedBounded:learn} \EndIfElse
                \State \Return $\texttt{NULL}$
            \Else
                \State \Return $\{ \langle n, c \rangle | n \in $ \Call{nodes}{$T$} $, c = \Call{sol}{n} \}$
            \EndIf
        \EndFunction
        \algstore{b2}
    \end{algorithmic}

    \begin{algorithmic}
        \algrestore{b2}
        \Function{verify}{$p, s, k, T, cand$}
            \For{$l \in \Call{leaves}{T}$}
            \State $\langle k', s'\rangle \gets $ \Call{outcome}{$s, k, cand, l$}
            \If{$p = \textsc{cont}$}
                \State $T' \gets \emptyset$
            \Else
                \State $T' \gets \{ cand(l) \}$
            \EndIf
                \State $a \gets $ \Call{solveAbstract}{\Call{opponent}{$p$}, $s'$, $k'$, $T'$}
                \IIf{$a \neq \texttt{NULL}$} \Return $\langle \True, l, a \rangle$ \EndIIf
            \EndFor
            \State \Return $\langle \False, \emptyset, \emptyset \rangle$
        \EndFunction
    \end{algorithmic}

    \caption{Unbounded realisability}
    \label{alg:unboundedBounded}
\end{algorithm}

%%%The states in an abstract game with no controller candidate are
%%%\emph{must-losing}. The environment can always force the game into the error
%%%states from these states.

%%%From abstractions with no environment candidate we record the complement of
%%%states in the tree as \emph{may-losing}. The environment cannot reach the error
%%%state in a number of steps equal to the distance to the bottom of the tree.

%%%We maintain a set of states for each rank up to the current bound. We maintain
%%%an invariant over these sets via careful construction so that they are
%%%monotonically increasing by rank. We also ensure that the environment is unable
%%%to force play from one set to the next. Due to these invariants, when two
%%%adjacent sets become equivalent we know that the algorithm has reached a fixed
%%%point and the controller is winning in the unbounded game (line 6).

We extend the bounded synthesis algorithm to learn states losing for one of the players from failed attempts to find candidate strategies.  The learning procedure kicks in whenever \textsc{findCandidate} cannot find a candidate strategy for an abstract game tree. When a player cannot win a abstract game of $k$ rounds from a state $s$ we have proven that $s$ is a losing state for any game with a height of $k$.  We can learn additional losing states from the tree via interpolation.  This is achieved in line~\ref{alg:unboundedBounded:learn} in Algorithm~\ref{alg:unboundedBounded}, enabled in the unbounded version of the algorithm, which invokes \textsc{learn} or \textsc{\textoverline{learn}} to learn controller or environment losing states respectively (Algorithm~\ref{alg:learn}).  

\subsection{Learning with interpolation}

Given two formulas $F_1$ and $F_2$ such that $F_1 \land F_2$ is unsatisfiable, it is possible to construct a Craig interpolant $\mathcal{I}$ such that $F_1 \to \mathcal{I}$, $F_2 \land \mathcal{I}$ is unsatisfiable, and $\mathcal{I}$ refers only to the intersection of variables in $F_1$ and $F_2$. 

We use these properties to learn the states that are losing to the actions in every subgame in an abstract game tree $T$ beginning in a state $s$. We assume that $T$ is labelled with controller actions and that $s$ is losing for the environment; learning controller losing states is described below.  We choose a non-leaf node $n$ of $T$ with maximal depth, i.e., a node whose children are leafs (Algorithm~\ref{alg:learn}, line~3), with the aim of learning the states that lose to the subgame beginning at $n$. Figure~\ref{fig:splita} shows a fragment of $T$ containing $n$.  First, we then split the tree at $n$ such that both slices $T_1$ and $T_2$ contain a copy of $n$ (line~4).  Figure~\ref{fig:splitb} shows $T_1$, which contains all of $T$ except the children of $n$, and $T_2$ (Figure~\ref{fig:splitc}), which contains only $n$ and its children.  There is no candidate strategy for $T$ so the formula constructed by unrolling copies of transition relation with fixed controller actions, $s \land \textsc{\textoverline{treeFormula}}(k, T)$, is unsatisfiable.  By construction, $ (\textsc{\textoverline{treeFormula}}(k, T_1) \land \textsc{\textoverline{treeFormula}}(k, T_2)) \implies \textsc{\textoverline{treeFormula}}(k, T)$ and so we know that $(s \land$ $\textsc{\textoverline{treeFormula}}(k, T_1) \land \textsc{\textoverline{treeFormula}}(k, T_2))$ is also unsatisfiable.  

\begin{figure}
    \begin{subfigure}[t]{0.3\textwidth}
        \centering
        \begin{tikzpicture}[sibling distance = 18mm, level distance = 14mm]
            \node [circle,draw] (root){}
                child {node [circle,draw] (n) {}
                    child {node [circle,draw,right=5pt] {}
                        edge from parent node [left=4pt] {$c_1$}
                    }
                    child {node [circle,draw,left=5pt] {}
                        edge from parent node [right=2pt] {$c_i$}
                    }
                    node [left=5pt] {$n$}
                }
                child {node [circle,draw] (tri0) {}
                    child {node [draw=none,right=5pt] (tri1) {}
                        edge from parent [draw=none]
                    }
                    child {node [draw=none,left=5pt] (tri2) {}
                        edge from parent [draw=none]
                    }
                }
                node [above=4pt] {$\vdots$};
            \draw (tri1.center) -- (tri2.center);
            \draw (tri0) -- (tri1.center);
            \draw (tri0) -- (tri2.center);
            \node [below=10pt of tri0] {$\ldots$};
            \node [below=10pt of root] {$\ldots$};
            \node [below=10pt of n] {$\ldots$};
        \end{tikzpicture}
        \caption{$T$}
        \label{fig:splita}
    \end{subfigure}%
    \begin{subfigure}[t]{0.3\textwidth}
        \centering
        \begin{tikzpicture}[sibling distance = 18mm, level distance = 14mm]
            \node [circle,draw] (root){}
                child {node [circle,draw] (n) {}
                    node [left=5pt] {$n$}
                }
                child {node [circle,draw] (tri0) {}
                    child {node [draw=none,right=5pt] (tri1) {}
                        edge from parent [draw=none]
                    }
                    child {node [draw=none,left=5pt] (tri2) {}
                        edge from parent [draw=none]
                    }
                }
                node [above=4pt] {$\vdots$};
            \draw (tri1.center) -- (tri2.center);
            \draw (tri0) -- (tri1.center);
            \draw (tri0) -- (tri2.center);
            \node [below=10pt of tri0] {$\ldots$};
            \node [below=10pt of root] {$\ldots$};
        \end{tikzpicture}
        \caption{$T_1$}
        \label{fig:splitb}
    \end{subfigure}%
    \begin{subfigure}[t]{0.3\textwidth}
        \centering
        \begin{tikzpicture}[sibling distance = 18mm, level distance = 14mm]
            \node [circle,draw] (n) {}
                child {node [circle,draw,right=5pt] {}
                    edge from parent node [left=4pt] {$c_1$}
                }
                child {node [circle,draw,left=5pt] {}
                    edge from parent node [right=2pt] {$c_i$}
                }
                node [left=5pt] {$n$};
            \node [below=10pt of n] {$\ldots$};
        \end{tikzpicture}
        \caption{$T_2$}
        \label{fig:splitc}
    \end{subfigure}
    \caption{Splitting a certificate tree}
    \label{fig:splitting}
\end{figure}

\begin{algorithm}
    \begin{algorithmic}[1]
        \Require $s(\mathcal{S}_T) \land $ \Call{treeFormula}{$k, T$} $\equiv \bot$
        \Require \emph{Must-invariant} holds
        \Ensure \emph{Must-invariant} holds
        \Ensure $s \land B^M \not\equiv \bot$
        \Comment $s$ will be added to $B^M$
        \Function{learn}{$s, T$}
            \IIf{\Call{succ}{$T$} $= \emptyset$}
                \Return
            \EndIIf
            \State $n \gets $ non-leaf node with min height 
            \State $\langle T_1, T_2 \rangle \gets $ \Call{gtSplit}{$T, n$}
            \State $F_1 \gets s(\mathcal{S}_T) \land \Call{treeFormula}{k, T_1}$
            \State $F_2 \gets \Call{treeFormula}{k, T_2}$
            \State $\mathcal{I} \gets  \Call{interpolate}{F_1, F_2}$
%%%            \If{\text{strategy generation enabled}}
%%%                \State $Strats^e \gets \Call{genLocalStrats}{\mathcal{I}, T_2}$\label{line:estrats}
%%%            \EndIf
            \State $B^M \gets B^M \lor \mathcal{I}$
            \State \Call{learn}{$s, T_1$}
        \EndFunction
        \algstore{learn}
    \end{algorithmic}
    
    \begin{algorithmic}[1]
        \algrestore{learn}
        \Require $s(\mathcal{S}_T) \land $ \Call{\textoverline{treeFormula}}{$k, T$} $\equiv \bot$
        \Require \emph{May-invariant} holds
        \Ensure \emph{May-invariant} holds
        \Ensure $s \land B^m[$\Call{height}{$k, T$}$] \equiv \bot$
        \Comment{$s$ will be removed from $B^m$}
        \Function{\textoverline{learn}}{$s, T$}
            \IIf{\Call{succ}{$T$} $= \emptyset$}
                \Return
            \EndIIf
            \State $n \gets $ non-leaf node with min height 
            \State $\langle T_1, T_2 \rangle \gets $ \Call{gtSplit}{$T, n$}
            \State $F_1 \gets s(\mathcal{S}_T) \land $ \Call{\textoverline{treeFormula}}{$k, T_1$}
            \State $F_2 \gets \Call{\textoverline{treeFormula}}{k, T_2}$
            \State $\mathcal{I} \gets $ \Call{interpolate}{$F_1, F_2$}
            \If{\text{strategy generation enabled}}
                \State $Strats \gets \Call{genLocalStrats}{\mathcal{I}, T_2}$\label{cstrats}
            \EndIf
            \For{$i = 1$ to \Call{height}{$k, n$}}
                \State $B^m[i] \gets B^m[i] \setminus \mathcal{I}$
            \EndFor
            \State \Call{\textoverline{learn}}{$s, T_1$}
        \EndFunction
    \end{algorithmic}
    \caption{Learning algorithms}
    \label{alg:learn}
\end{algorithm}

%%%This enables the construction of an interpolant that captures losing states at
%%%node $n$.

We construct an interpolant with $F_1 = s(\mathcal{S}_T) \land \textsc{treeFormula}(k, T_1)$ and $F_2 = \textsc{treeFormula}(k, T_2)$ (line~5). The only variables shared between $F_1$ and $F_2$ are the state variable copies belonging to node $n$. By the properties of the interpolant, $F_2 \land \mathcal{I}$ is unsatisfiable, therefore all states in $\mathcal{I}$ are losing against abstract game tree $T_2$ in Figure~\ref{fig:splitc}.  We also know that $F_1 \to \mathcal{I}$, thus $\mathcal{I}$ contains all states reachable at $n$ by following $T_1$ and avoiding error states.  

We have discovered a set $\mathcal{I}$ of states losing for the environment and now must record it. Care must be taken in this step because the learning procedure operates on a bounded game. We have learned that there is no environment strategy that forces the game into an error state from $\mathcal{I}$ in $k$ rounds or less but there may be a longer strategy that does force an error.  We therefore record learned environment-losing states along with associated bounds.  To this end, we maintain a conceptually infinite array of sets $B^m[k]$ that are may-losing for the controller, indexed by bound $k$.  $B^m[k]$ are initialised to $\top$ for all $k>0$ and $B^m[0] \gets E$.  Whenever an environment-losing set $\mathcal{I}$ is discovered for a node $n$ with bound $\textsc{height}(k, n)$ in line~13 of Algorithm~\ref{alg:learn}, this set is subtracted from $B^m[i]$, for all $i$ less than or equal to the bound (lines~14--16).

Learning of states losing for the controller is similar (\textsc{learn} in Algorithm~\ref{alg:learn}). The main difference is that environment-losing states are losing for all game heights. Therefore we record these states in a single set $B^M$ of must-losing states (Algorithm~\ref{alg:learn}, line~6).  This set is initialised to the error set $E$ and grows as new losing states are discovered.  

Once the set of states losing for the subtree $T_2$ has been recorded we continue to learn states from the other nodes in $T_1$. This achieved by recursively splitting the already reduced $T_1$ and learning the states that lose to a different subtree. In order to compute interpolants we have to ensure that the formulas constructed from trees during recursive calls are unsatisfiable. For controller losing trees we may have removed a subtree in which the environment forces a visit to an error state and enabled a satisfying assignment to the SAT query. To handle this case we introduce constraints to \textsc{treeFormula} (and similarly \textoverline{\textsc{treeFormula}}) to take into account previously learned states. Intuitively, we capture $T_2$ by forbidding the player from visiting the states that lose to the actions in the tree. As a result any pairs of formulas constructed during the recursive decomposition of $T$ are unsatisfiable and enable interpolation.

\subsection{Example}

Consider a simple arbiter system in which the environment makes a request for a number of resources (1 or 2), and the controller may grant access to up to two resources.  The total number of requests grows each round by the number of environment requests and shrinks by the number of resources granted by the controller in the previous round.  The controller must ensure that the number of unhandled requests does not accumulate to more than 2.  Figure~\ref{fig:example} shows the variables (\ref{fig:examplevars}), the initial state of the system (\ref{fig:exampleinit}), and the formulas for computing next-state variable assignments (\ref{fig:exampletrans}) for this example. We use primed identifiers to denote next-state variables and curly braces to define the domain of a variable.


\begin{figure}
    \begin{subfigure}[t]{\textwidth}
        \centering
        \begin{tabular}{l | l | l}
            \textbf{Uncontrollable} & \textbf{Controllable} & \textbf{State} \\
            \hline
            \texttt{request = \{1,2\}} & \texttt{grant0 = \{0,1\}} & \texttt{resource0 = \{0,1\}} \\
            & \texttt{grant1 = \{0,1\}} & \texttt{resource1 = \{0,1\}} \\
            & & \texttt{nrequests = \{0,1,2,3\}} \\
        \end{tabular}
        \caption{Variables}
        \label{fig:examplevars}
    \end{subfigure}

    \vspace{5mm}
    \begin{subfigure}[t]{\textwidth}
        \centering
        \texttt{resource0 = 0; resource1 = 0; nrequests = 0;}
        \caption{Initial State}
        \label{fig:exampleinit}
    \end{subfigure}

    \begin{subfigure}[t]{\textwidth}
        \begin {align*}
            \texttt{resource0'} & \texttt{ = grant0;} \\
            \texttt{resource1'} & \texttt{ = grant1;} \\
            \texttt{nrequests'} & \texttt{ = (nrequests + request >= resource0 + resource1)} \\ 
                                & \texttt{ ? (nrequests + request - resource0 - resource1)}\\
                                & \texttt{ : 0;}
        \end{align*}
        \caption{Transition Relation}
        \label{fig:exampletrans}
    \end{subfigure}
    \caption{Example}
    \label{fig:example}
\end{figure}


\begin{figure}[t]
    \centering
    \begin{subfigure}[t]{.32\textwidth}
        \centering

        \begin{tikzpicture}[sibling distance = 18mm, level distance = 14mm]
            \node [circle,draw] (root){}
                child {node [circle,draw] {}
                    child {node [circle,draw,right=5pt] {}
                        edge from parent node [left=4pt,text width=1cm] {\texttt{gr0=1 gr1=0}}
                    }
                    child {node [circle,draw,left=5pt] {}
                        edge from parent node [right=2pt,text width=1cm] {\texttt{gr0=1 gr1=1}}
                    }
                    node [left=5pt] {$n$}
                    edge from parent node [above left=-6pt and 2pt,text width=1cm] {\texttt{gr0=0 gr1=1}}
                }
                child {node [circle,draw] (n) {}
                    child {node [circle,draw] {}
                        edge from parent node [right=2pt,text width=1cm] {\texttt{gr0=1 gr1=1}}
                    }
                    edge from parent node [above right=-6pt and 2pt,text width=1cm] {\texttt{gr0=1 gr1=1}}
                }
                node [above=4pt] {$\langle s, k \rangle$};
        \end{tikzpicture}
        \caption{A losing AGT $T$}
        \label{fig:interpolatetree}
    \end{subfigure}

    \begin{subfigure}[t]{.5\textwidth}
        \centering
        \begin{tikzpicture}[sibling distance = 18mm, level distance = 14mm]
            \node [circle,draw] (root){}
                child {node [circle,draw] {}
                    node [left=5pt] {$n$}
                    edge from parent node [above left=-6pt and 2pt,text width=1cm] {\texttt{gr0=1 gr1=0}}
                }
                child {node [circle,draw] (n) {}
                    child {node [circle,draw] {}
                        edge from parent node [right=2pt,text width=1cm] {\texttt{gr0=1 gr1=1}}
                    }
                    edge from parent node [above right=-6pt and 2pt,text width=1cm] {\texttt{gr0=1 gr1=1}}
                }
                node [above=4pt] {$\langle s, k \rangle$};
        \end{tikzpicture}
        \caption{Tree slice $T_1$}
        \label{fig:treef1}
    \end{subfigure}%
    \begin{subfigure}[t]{.5\textwidth}
        \centering
        \begin{tikzpicture}[sibling distance = 18mm, level distance = 14mm]
            \node [circle,draw] {}
                child {node [circle,draw,right=5pt] {}
                    edge from parent node [left=6pt,text width=1cm] {\texttt{gr0=1 gr1=0}}
                }
                child {node [circle,draw,left=5pt] {}
                    edge from parent node [right=6pt,text width=1cm] {\texttt{gr0=1 gr1=1}}
                }
                node [left=5pt] {$n$};
        \end{tikzpicture}
        \caption{Tree slice $T_2$}
        \label{fig:treef2}
    \end{subfigure}
    \caption{Splitting of an abstract game tree by the learning procedure.}
    \label{fig:interpolanttrees}
\end{figure}




Consider node $n$ in Figure~\ref{fig:interpolatetree}, which shows an abstract game tree for which the environment has no winning action. At this node there are two controller actions that prevent the environment from forcing the game into an error state in one game round. We want to use this tree to learn the states from which the controller can win playing one of these actions. Specifically, we compute a subset of such states, using interpolation, which we will show is sufficient to ensure convergence.

We construct an interpolant with $F_1 = s(\mathcal{S}_T) \land \textsc{treeFormula}(k, T_1)$ and $F_2 = \textsc{treeFormula}(k, T_2)$ (line~5). The only variables shared between $F_1$ and $F_2$ are the state variable copies belonging to node $n$. By the properties of the interpolant, $F_2 \land \mathcal{I}$ is unsatisfiable, therefore all states in $\mathcal{I}$ are losing against abstract game tree $T_2$ in Figure~\ref{fig:treef2}.  We also know that $F_1 \to \mathcal{I}$, thus $\mathcal{I}$ contains all states reachable at $n$ by following $T_1$ and avoiding error states.  

At node $n$, the interpolant $(\texttt{nrequests} = 1 \land
\texttt{resource1} = 1)$ captures the information we need. Any action by the
environment followed by one of the controller actions at $n$ will be
winning for the controller.

\begin{algorithm}[t] \caption{Tree formula construction with $B^m$ and $B^M$} \label{alg:unboundedTreeFormula} \begin{algorithmic}[1]
        \Function{treeFormula}{$k, T$} \If{$\Call{height}{k, T} = 0$} \State
        \Return{ $\lnot \Call{$B^M$}{\mathcal{S}_{T}}$ } \Else \State \Return{$\lnot
            \Call{$B^M$}{\mathcal{S}_{T}} \land$ \\ $$\bigwedge_{\langle e, n \rangle \in
            \Call{succ}{T}}(\Call{$\delta$}{\cS_T, \cU_e, \cC_n, \cS_n} \land \cU_e =
            \Call{action}{e} \land \Call{treeFormula}{k, n})$$ } \EndIf
        \EndFunction \algstore{tf1} \end{algorithmic}

    \begin{algorithmic}[1]
        \algrestore{tf1}
        \Function{\textoverline{treeFormula}}{$k, T$}
        \If{$\Call{height}{k, T} = 0$}
        \State \Return{\Call{E}{$\mathcal{S}_T$}}
        \Else
        \State \Return{ \Call{$B^m[\Call{height}{k, T}]$}{$\mathcal{S}_{T}$} $\land$ \\
            $$\bigg( \Call{E}{\mathcal{S}_T} \lor \bigvee_{\langle e, n \rangle \in \Call{succ}{T}}(\Call{$\delta$}{\cS_T, \cU_n, \cC_e, \cS_n} \land \cC_e = \Call{action}{e} \land \Call{\textoverline{treeFormula}}{k, n})\bigg)$$ }
        \EndIf
        \EndFunction
    \end{algorithmic}
\end{algorithm}

\subsection{Convergence on a fixed point}

Algorithm~\ref{alg:unbounded} shows the main loop of the unbounded synthesis algorithm.  The algorithm invokes the modified bounded synthesis procedure with increasing bound $k$ until the initial state is in $B^M$ (environment wins) or $B^m$ reaches a fixed point (controller wins). We now prove that the algorithm will eventually converge on a fixed point and that it guarantees that the game is realisable.

\begin{algorithm}[h]
    \begin{algorithmic}[1]
        \Function{solveUnbounded}{$G, E$}
            \State $B^M \gets E$
            \State $B^m[0] \gets E$
            \For{$k = 1 \dots$}
                \If{\Call{SAT}{$s_0 \land B^M$}}
                    \LineComment{Losing in the initial state}
                    \State \Return \texttt{unrealisable} 
                \EndIf
                \If{$\exists i < k . \  B^m[i] \equiv B^m[i+1]$}
                    \LineComment{Reached fixed point}
                    \State \Return \texttt{realisable} 
                \EndIf
                \State $B^m[k] \gets \top$
                \State \Call{checkBound}{$k$}
            \EndFor
        \EndFunction
        \algstore{u1}
    \end{algorithmic}

    \begin{algorithmic}
        \algrestore{u1}
        \Require \emph{May} and \emph{must} invariants hold
        \Ensure \emph{May} and \emph{must} invariants hold
        \Ensure $s_0 \not\in B^m[k]$ if there exists a winning controller strategy with bound $k$
        \Ensure $s_0 \in B^M$ if there exists a winning environment strategy with bound $k$
%%%        $\exists u_{k..1} \forall c_{k..1} \  s(x_k) \land (E_k \lor (\delta(x_k, u_k, c_k) \land E(x_{k-1}) \lor ... \land E(x_1) \lor (\delta(x_1, u_1, c_1) \land E(x_0))...)$
%%%            $\forall u_{k..1} \exists c_{k..1} \  s(x_k) \land \lnot E_k \land \delta(x_k, u_k, c_k) \land \lnot E(x_{k-1}) \land ... \land \delta(x_1, u_1, c_1) \land \lnot E(x_0)$
        \Function{checkBound}{$k$}
            \State \Return \Call{solveAbstract}{$\texttt{env}, s_0, k, \emptyset$}
        \EndFunction
    \end{algorithmic}
    \caption{Unbounded Synthesis}
    \label{alg:unbounded}
\end{algorithm}

We define two global invariants of the algorithm.  The \emph{may-invariant}
states that sets $B^m[i]$ decrease monotonically with $i$ and that each $B^m[i+1]$ overapproximates the states from which the environment can force the game into $B^m[i]$, i.e. $B^m[i+1]$ overapproximates the uncontrollable predecessor of $B^m[i]$.  The \emph{must-invariant} guarantees that the must-losing set $B^M$ is an underapproximation of the actual losing set $B$.  Both invariants trivially hold after $B^m$ and $B^M$ have been initialised in the beginning of the algorithm.  Further interaction of the algorithm with the recorded learned states is constrained to \textsc{learn} and \textoverline{\textsc{learn}} so it suffices to prove these invariants for these functions only.

\begin{proposition}\label{prop:opplearn}
    Assuming the preconditions are met, \textoverline{\textsc{learn}} satisfies its postconditions. Namely,
    \begin{enumerate}
        \item \textoverline{\textsc{learn}} maintains the may-invariant: $$\forall (0<i<k).~B^m[i] \subseteq B^m[i+1], Upre(B^m[i]) \subseteq B^m[i+1].$$
        \item \textoverline{\textsc{learn}} ensures that $s$ is removed from $B^m$: $$s \land B^m[\textsc{height}(k, T)] \equiv \bot$$
    \end{enumerate}
\end{proposition}
\begin{proof}
    The precondition of \textoverline{\textsc{learn}} states that $s(\cS_T) \land \textsc{\textoverline{treeFormula}}(k, T) \equiv \bot$. We use the unsatisfiability of this formula to construct an interpolant.  Line~(11--12) splits the tree $T$ into $T_1$ and $T_2$, such that $T_2$ has depth 1.  Consider formulas $F_1=s(\cS_T) \land \textsc{\textoverline{treeFormula}}(k, T_1)$ and $F_2 = \textsc{\textoverline{treeFormula}}(k, T_2)$.  These formulas only share variables $\mathcal{S}_n$.  Their conjunction $F_1 \land F_2$ is unsatisfiable, as by construction any solution of $F_1 \land F_2$ also satisfies $s(\cS_T) \land \textsc{\textoverline{treeFormula}}(k, T)$, which is unsatisfiable by the precondition.  Hence the interpolation operation is defined for $F_1$ and $F_2$.  

Intuitively, the interpolant computed in line~(13) overapproximates the set of states reachable from $s$ by following the tree from the root node to $n$, and underapproximates the set of states from which the environment loses against tree $T_2$.  

Formally, $\II$ has the property $\II \land F_2 \equiv \bot$.  Since $T_2$ is of depth 1, this means that the environment cannot force the game into $B^m[\textsc{height}(k, n)-1]$ playing against the counterexample moves in $T_2$.  Hence, $\II \cap Upre(B^m[\textsc{height}(k, n)-1]) = \emptyset$.  Furthermore, since the may-invariant holds, $\II \cap Upre(B^m[i]) = \emptyset$, for all $i < \textsc{height}(k, n)$.  Hence, removing $\II$ from all $B^m[i], i\leq \textsc{height}(k, n)$ in line~(15) preserves the may-invariant, thus satisfying the first post-condition.

Furthermore, the interpolant satisfies $F_1 \rightarrow \II$, i.e., any assignment to $\mathcal{S}_n$ that satisfies $s(\mathcal{S}_T) \land \textsc{\textoverline{treeFormula}}(k, T_1)$ also satisfies $\II$.  Hence, removing $\II$ from $B^m[\textsc{height}(k, n)]$ makes $s(\mathcal{S}_T) \land \textsc{\textoverline{treeFormula}}(k, T_1)$ unsatisfiable, and hence all preconditions of the recursive invocation of \textsc{\textoverline{learn}} in line~(17) are satisfied.  

At the second last recursive call to \textsc{\textoverline{learn}}, tree $T_1$ is empty, $n$ is the root node, $\textsc{\textoverline{treeFormula}}(k, T_1) \equiv B^m[\textsc{height}(k, T_1)](\mathcal{S}^T)$; hence $s(\mathcal{S}_T) \land \textsc{\textoverline{treeFormula}}(k, T_1) \equiv s(\mathcal{S}_T) \land B^m[\textsc{height}(k, T_1)](\mathcal{S}^T) \equiv \bot$.  Thus the second postcondition of \textsc{\textoverline{learn}} holds.

\end{proof}

\begin{proposition}\label{prop:learn}
    Assuming the preconditions are met, \textsc{learn} satisfies its postconditions. Namely,
    \begin{enumerate}
        \item \textsc{learn} maintains the \emph{must-invariant}: $$B^M \subseteq B.$$
        \item \textsc{learn} ensures that $s$ is added to $B^M$: $$s \land B^M \not\equiv$$
    \end{enumerate}
\end{proposition}

\begin{proof}
The proof of \textsc{learn} is similar to the above proof of \textsc{\textoverline{learn}}. An interpolant constructed from $F_1=s(\mathcal{S}_T) \land \textsc{treeFormula}(k, T_1)$ and $F_2 = \textsc{treeFormula}(k, T_2)$ has the property $\II \land F_2 \equiv \bot$ and the precondition ensures that the controller is unable to force the game into $B^M$ playing against the counterexample moves in $T_2$. Thus adding $\II$ to $B^M$ maintains the must-invariant satisfying the first postcondition.  

Likewise, in the second last recursive call of \textsc{learn} with the empty tree $T_1$ and root node $n$: $\textsc{treeFormula}(k, T_1) \equiv \lnot B^M(\mathcal{S}_T)$.  Hence $s(\mathcal{S}_T) \land \textsc{treeFormula}(k, T_1) \equiv s(\mathcal{S}_T) \land \lnot B^M(\mathcal{S}_T) \equiv \bot$. Therefore $s \land B^M \not\equiv \bot$, the second postcondition, is true.
\end{proof}

\begin{proposition}\label{prop:checkBound}
    We prove the following for \textsc{checkBound}
    \begin{enumerate}
        \item \textsc{checkBound} maintains the may and must invariants
        \item If the bounded game is realisable for $k$ then \textsc{checkBound} terminates and guarantees $s_0 \not\in B^m[k]$.
        \item If the bounded game is unrealisable for $k$ then \textsc{checkBound} terminates and guarantees $s_0 \in B^M$.
    \end{enumerate}
\end{proposition}
\begin{proof}
    $B^m$ and $B^M$ are updated inside $\textsc{\textoverline{learn}}$ and $\textsc{learn}$ only. Therefore, by Propositions~\ref{prop:opplearn} and \ref{prop:learn}, \textsc{checkBound} maintains both invariants.

    From the correctness of \textsc{solveAbstract} given in Theorem~\ref{theorem:solveAbstract} we ensure that \textsc{checkBound} terminates and during execution a certificate tree is generated and checked by \textsc{findCandidate}. In the modified version of \textsc{solveAbstract}, one of the learning procedures is called when \textsc{findCandidate} returns \texttt{NULL}. Thus \textsc{checkBound} generates a call to a learning procedure and passes $s$ and a certificate tree $T$.  Therefore, when the game is realisable, Proposition~\ref{prop:opplearn} guarantees that $s$ is removed from $B^m[k]$. Likewise, when the game is unrealisable Proposition~\ref{prop:learn} guarantees that $s$ is removed from $B^M$.

%%%Assume that there is a winning strategy for the controller at bound $\kappa$.  This means that at some point the algorithm discovers a counterexample tree of bound $\kappa$ for which the environment cannot force into $E$. The algorithm then invokes the \textsc{\textoverline{learn}} method, which removes $s_0$ from $B^m[\kappa]$.  Alternatively, if there is a winning strategy for the environment at bound $\kappa$ then a counterexample losing for the controller will be found.  Subsequently \textsc{learn} will be called and $s_0$ added to $B^M$.
\end{proof}


\begin{theorem}
    Let $G$ be a game with a safety condition defined by a set of error states $E$.  $\textsc{solveUnbounded}(G, E)$ is guaranteed to terminate and correctly decide realisability for $G$.
\end{theorem}
\begin{proof}
Assume that $G$ is realisable, we use the may-invariant to show correctness of \textsc{solveUnbounded}.  The invariant guarantees that $B^m[i]$ contains all states from which the environment can force the game into an error state in at most $i$ steps. If $B^m[i] \equiv B^m[i+1]$ (line~6) then the invariant states that $Upre(B^m[i]) \subseteq B^m[i]$. Thus, any state that the environment can use to force the game into $B^m[i]$ is contained within $B^m[i]$. In other words, $B^m[i]$ overapproximates the winning states for the environment.  Proposition~\ref{prop:checkBound} ensures that $s_0 \not\in B^m[k]$, and since the invariant ensures that $B^m$ is monotonic then $s_0$ must not be in $B^m[i]$. If $s_0 \not\in B^m[i]$ then $s_0$ is not in the winning states for the environment and the controller can always win from $s_0$.  

Alternatively, $G$ is unrealisable. The must-invariant guarantees that the environment can force the game into an error state from $B^M$, therefore checking whether the initial state is in $B^M$ (as in line~5) is sufficient to return \texttt{unrealisable}.  

Termination also follows from the invariants. Given that $B^m$ decreases monotonically and there are a finite set of states in the game, the algorithm is guaranteed to reach a fixed point if the game is realisable. If the game is unrealisable there must exist a finite length winning environment strategy, thus by iteratively increasing $k$ we guarantee termination.

\end{proof}

\subsection{Strategy extraction}

In the previous chapter I showed how to extract a strategy from the certificate tree generated during the bounded realisability algorithm. I will now prove that a similar approach extracts a correct strategy for an unbounded game when strategy generation is done online as it is when the learning optimisation is enabled for bounded realisability. In order to extract strategies online, line~\ref{line:cstrats} becomes active in \textoverline{\textsc{learn}} and a local strategy is generated for each learned interpolant.

\begin{theorem}
    Let $G$ be a safety game that is realisable with respect to an error set $E$. Let $Strats$ be a set of partial strategies generated during $\textsc{solveUnbounded}(G, E)$, then $\pi = \textsc{compileStrat}(Strats)$ is a winning partial controller strategy for $G$.
\end{theorem}
\begin{proof}
    Let $(b, k)$ be a pair such that $b$ is a set of states learned to be environment-losing for a game of length $k$ during execution of \textoverline{\textsc{learn}}. Then $(b, k)$ has corresponding tree with a depth of one $T$. We record $[(w_0, c_0, k_0), \ldots, (w_j, c_j, k_j)] = \textsc{genLocalStrats}(T, k, b)$ using \opptf from Algorithm~\ref{alg:unboundedTreeFormula}. Since the tree is of depth one, \textsc{next} is never called.  From the precondition of \textoverline{\textsc{learn}} we know that $b(\cS_T) \land \opptf(k, T)$ is false. Therefore interpolation in \textsc{partition} is well defined. The resulting partitioning defines a controller action $c_i$ such that every $c_i$-successor of the set of states and environment actions in $w_i$ is outside $B^m[k_i-1]$. By construction $b(\cW) = \bigvee_i w_i$, i.e. the partial strategy defines an action for every state in $b$ and every environment action.

    On termination of unbounded realisability $s_0$ will be removed from $B^m[\kappa]$. By collecting all the partial strategies generated online we have a controller strategy from $s_0$ that ensures that the environment can never reach any set in $B^m$. Since $B^m[i] = B^m[i+1]$ is known to overapproximate the winning region of the environment, the resulting strategy must stay within the safe region.
\end{proof}

\section{Optimisations}

\subsection{Generalising the initial state}

This optimisation allows us to learn may and must losing states faster.
Starting with a larger set of initial states we increase the reachable set and
hence increase the number of states learned by interpolation. This optimisation
requires a modification to $\textsc{solveAbstract}$ to handle sets of states,
which is not shown.

The optimisation is relatively simple and is inspired by a common greedy
heuristic for minimising $\texttt{unsat}$ cores. Initial state $s_0$ assigns a value to
each variable in $\mathcal{S}$. If the environment loses $\langle s_0, k
\rangle$ then we attempt to solve for a generalised version of $s_0$ by removing
one variable assignment at a time. If the environment loses from the larger set of
states then we continue generalising. In this way we learn more
states by increasing the reachable set. In our benchmarks we have observed that
this optimisation is beneficial on the first few iterations of
\textsc{checkBound}.

\begin{algorithm}
    \begin{algorithmic}
        \Function{checkBound}{$k$}
            \State $r \gets $ \Call{solveAbstract}{$\texttt{env}, s_0, k, \emptyset$}
            \IIf{$r \neq \emptyset$} \Return $r$ \EndIIf
            \State $s' \gets s_0$
            \For{$x \in \mathcal{S}$}
            \State $r \gets$ \Call{solveAbstract}{$\texttt{env}, s' \setminus \{x\}, k, \emptyset$} 
                \If{$r = \texttt{NULL}$}
                    \LineComment{Remove the assignment to $x$ from $s'$}
                    \State $s' \gets s' \setminus \{x\}$ 
                \EndIf 
            \EndFor
            \State \Return $\texttt{NULL}$
        \EndFunction
    \end{algorithmic}
    \caption{Generalise $s_0$ optimisation}
    \label{alg:opt1}
\end{algorithm}

\subsection{Generalising losing states}

The same generalisation operation can be performed during computational learning. When learning with interpolation the size of the learned set can vary from exactly the reachable states to a large overapproximation of reachable states. The interpolant is constructed with the property $F_1 \implies \mathcal{I}$, so by increasing the number of states represented by $F_1$ we also increase the size of $\mathcal{I}$. $F_1$ is given by $s(\mathcal{S}_{T_1}) \land \opptf(T_1, k)$ and represents the set of states reachable from $s$ by playing the actions in $T_1$. For correctness we require that $s$ is losing to the full tree $T$ so we can increase the size of $s$ by generalising in the same way as the previous optimisation. We drop variables from $s$ such that the new cube $s'$ is also losing to the actions in $T$. This increases $F_1$ and we may learn a larger interpolant.

\subsection{Improving candidate strategies}

In Chapter~\ref{ch:bounded} I introduced strategy shortening to improve candidate strategies. For unbounded synthesis this optimisations is even more useful because it allows the algorithm to learn losing states more quickly. Strategy shortening attempts to push \emph{good} actions closer to the root of the game tree. For learning environment-losing states this can mean that states can be proven to be losing for larger values of $k$. 

As long as $\textsc{findCandidate}$ returns a strategy whenever one exists there is significant freedom for heuristics to choose strategies without affecting correctness. For example, an additional SAT call in $\textsc{findCandidate}$ can first check for the existence of a strategy that loops by adding the requirement that $\exists i \exists j (s_i = s_j)$ in all runs $s_0, ..., s_k$. The motivation behind this optimisation was that a strategy that forces a loop will quickly converge on a fixed point. In practice, such strategies are rarely found and the additional call to the SAT solver was a waste of resources. It remains future work to fully explore the possibilities of heuristically selecting candidates. One avenue for exploration is the use of QBF or 2QBF solvers to check for actions that are winning for all opponent actions, i.e. partially reintroducing universal quantifiers to the search.

\section{Discussion}

Unbounded realisability is designed to take advantage of the strengths of bounded realisability but provide the completeness offered by a fixed point computation. These conflicting aims are addressed by overapproximation via interpolation, which allows completeness without sacrificing performance to an exploding symbolic representation.

\subsection{Related work}

Synthesis of safety games is a thoroughly explored area of research with most
efforts directed toward solving games with BDDs \cite{Burch90} and abstract
interpretation \cite{Walker14,Brenguier14}. Satisfiability solving has been used
previously for synthesis in a suite of methods proposed by Bloem et al.
\cite{Bloem14}. The authors propose employing competing SAT solvers to learn
clauses representing bad states, which is similar to our approach but does not
unroll the game.  They also suggest QBF solver, template-based, and Effectively
Propositional Logic (EPR) approaches.

SAT-based bounded model checking approaches that unroll the transition relation
have been extended to unbounded by using conflicts in the solver
\cite{McMillan02}, or by interpolation \cite{McMillan03}. However, there
are no corresponding adaptations to synthesis.

Incremental induction \cite{Bradley11} is another technique for unbounded
model checking that inspired several approaches to synthesis including the work
presented here.  Morgenstern et al.~\cite{Morgenstern13} proposed an
technique that computes sets of states that overapproximate the losing states
(similar to our $B^m$) and another set of winning states (similar to the
negation of $B^M$).  Their algorithm maintains a similar invariant over the
sets of losing states as our approach and has the same termination condition.
It differs in how the sets are computed, which it does by inductively proving
the number of game rounds required by the environment to win from a state.
Chiang and Jiang~\cite{Chiang15} recently proposed a similar approach that
focusses on computing the winning region for the controller forwards from the
initial state in order to take advantage of reachability information and bad
transition learning without needing to discard learnt clauses.

There are approaches to synthesis of LTL specifications that use bounds to simplify the problem.  The authors of \cite{Finkbeiner13} suggest a methodology directly inspired by bounded model checking and it has been adapted to symbolic synthesis \cite{Ehlers12}. In contrast to the approach here the bound is placed on the implementation instead of the number of game rounds.  Lazy synthesis \cite{Finkbeiner12} similarly constructs implementations of a bounded size but does so in a counterexample guided approach. Their approach is to use an SMT solver to produce a candidate implementation and then check the implementation with a BDD based model checker. These bounded synthesis techniques are similar in idealogy to the approach described here but are used to solve a different problem.

%%%The original bounded synthesis algorithm of Narodytska et
%%%al.~\cite{narodytska14} solves realisability without constructing a strategy.
%%%In \cite{een2015} the realisability algorithm is extended with strategy
%%%extraction. The technique relies on interpolation over abstract game trees to
%%%compute the winning strategy.  In the present work we use interpolation in a
%%%different way in order to learn losing states of the game. In addition, this
%%%method could be easily adapted to the unbounded realisability algorithm
%%%presented here to generate unbounded strategies.

\subsection{Limitations}

Bounded synthesis is generally efficient for games without a high branching factor, as discussed in Chapter~\ref{ch:bounded}. This limitation affects the unbounded solver as can be seen in the synthesis competition results on specifications such as the adder. In a correct adder the controller must set a variable $c$ to be equal to $a + b$ where $a$ and $b$ are environment variables. The unbounded synthesis algorithm must construct a game tree consisting of all possible values to $c$ in order to prove realisability.

The unbounded solver extends the bounded solver with learning. Learning states from the game tree does not introduce significant complexity to the problem. However, there are cases in which learning can be slow to converge on a fixed point. As a result the bounded algorithm must be iterated many times with increasing bounds.

\begin{figure}
    \begin{subfigure}[t]{\textwidth}
        \centering
        \begin{tabular}{l | l | l}
            \textbf{Uncontrollable} & \textbf{Controllable} & \textbf{State} \\
            \hline
            \texttt{stay = \{0,1\}} & \texttt{reset = \{0,1\}} & \texttt{counter = \{0,..,2\textsuperscript{n}-1\}} \\
                                    & & \texttt{err = \{0,1\}}
        \end{tabular}
        \caption{Variables}
        \label{fig:examplevars}
    \end{subfigure}

    \vspace{5mm}
    \begin{subfigure}[t]{\textwidth}
        \centering
        \texttt{counter = 0; err = 0;}
        \caption{Initial State}
        \label{fig:exampleinit}
    \end{subfigure}

    \begin{subfigure}[t]{\textwidth}
        \begin {align*}
        \texttt{counter' = }{}& \texttt{if (stay)} \\
        {}& \texttt{\ \ counter} \\
        {}& \texttt{else if (counter == (2\textsuperscript{n-1})-1 } \land \texttt{ reset)} \\
        {}& \texttt{\ \ 0} \\
        {}& \texttt{else} \\
        {}& \texttt{\ \ counter + 1} \\
        \texttt{err' = }{}& \texttt{(counter == 2\textsuperscript{n}-1)}
        \end{align*}
        \caption{Transition Relation}
        \label{fig:exampletrans}
    \end{subfigure}
    \caption{Parameterised counter example}
    \label{fig:counter}
\end{figure}

In the synthesis competition benchmarks there is a simple counter specification that helps illustrate the limitations of unbounded synthesis. The specification is given a parameter $n$ that determines the number of bits in the counter. Figure~\ref{fig:counter} shows the specification. The environment has the choice to increment the counter, or not. The controller can reset the counter when it is half way to the maximum value. The controller is safe if the counter never reaches its maximum value. Clearly a safe controller resets whenever it is able to.

This example can produce several different interesting behaviours in the unbounded synthesis algorithm. Let us first consider the game tree in Figure~\ref{fig:limitationAGTa}. The controller strategy shown in this tree is to not reset the counter. This strategy wins the bounded game for any counter with $n \geq 2$ because there are not enough game rounds for the counter to reach the error state. Let's now consider how the unbounded algorithm would learn from this tree when $n = 3$. At the node of height 1, the largest interpolant that could be learned is $((\texttt{err == 0}) \land (\texttt{counter < 6}))$. At the next node up the tree we might learn $((\texttt{err == 0}) \land (\texttt{counter < 5}))$. With a tree of length $k = 2^n - 2 = 6$ we could have $B^m$ with a difference of exactly one state in each successive index. For high values of $n$ it is infeasible to construct a formula in CNF for this many unrollings of the transition relation, even though there is no branching.

In practise, unbounded synthesis can do well on this specification when the SAT solver gives a candidate such as in Figure~\ref{fig:limitationAGTb}. The optimisation that generalises the losing state before interpolation is able to drop all bits of the counter except the high bit. This gives $((\texttt{err = 0}) \land (\texttt{counter < 3}))$, which describes the safe region of the game. The procedure quickly converges on this fixed point in this case.

\begin{figure}
    \begin{subfigure}[t]{0.5\textwidth}
        \centering
        \begin{tikzpicture}[sibling distance = 18mm, level distance = 14mm]
            \node [circle,draw] (root){}
                child {node [circle,draw] {}
                    child {node [circle,draw] {}
                        edge from parent node [right] {\texttt{reset=0}}
                    }
                    edge from parent node [right] {\texttt{reset=0}}
                }
                node [above=4pt] {$\langle s_0, 2 \rangle$};
        \end{tikzpicture}
        \caption{No reset candidate}
        \label{fig:limitationAGTa}
    \end{subfigure}
    \begin{subfigure}[t]{0.5\textwidth}
        \centering
        \begin{tikzpicture}[sibling distance = 18mm, level distance = 14mm]
            \node [circle,draw] (root){}
                child {node [circle,draw] {}
                    child {node [circle,draw] {}
                        edge from parent node [right] {\texttt{reset=1}}
                    }
                    edge from parent node [right] {\texttt{reset=1}}
                }
                node [above=4pt] {$\langle s_0, 2 \rangle$};
        \end{tikzpicture}
        \caption{Reset candidate}
        \label{fig:limitationAGTb}
    \end{subfigure}
    \caption{Game trees for the counter specification}
    \label{fig:limitationAGT}
\end{figure}

\subsection{Strengths}

The strength of unbounded synthesis lies in the ability to approximate the winning region. This can be seen on a scaled up version of the example used in Section~\ref{sec:unboundedAlgorithm}.  The example as described is the $n=2, m=2$ instance of the more general problem of an arbiter of $n$ resources and an environment that can make up to $m$ requests at once. The general problem is realisable if $n \geq m$ by a controller that always grants access to $m$ resources.  The problem for BDD solvers is that it can be difficult to compactly represent all $^n C_m$ combinations of granted resources that make up the winning region. With unbounded synthesis a fixed point can be reached by choosing just one winning combination and proving that the environment cannot force an escape from that small set of states.

%%%Consider the $n=2, m=2$ instance as given above.  One approach taken by BDD solvers is to iteratively compute the uncontrollable predecessor of the set of error states until a fixed point.  One iteration of this operation would give the set: $\texttt{nrequests} = 3 \lor (\texttt{nrequests} = 2 \land ( \texttt{resource0} = 0 \lor \texttt{resource1} = 0)) \lor (\texttt{nrequests} = 1 \land (\texttt{resource0} = 0 \land \texttt{resource1} = 0))$. 

%%%We try to avoid computing the entire winning set by employing interpolation to approximate it instead. The benefit of interpolation is that an approximation can be obtained efficiently from the resolution proof of two mutually unsatisfiability formulas.

In Chapter~\ref{ch:bounded} I showed that the strength of the bounded realisability algorithm lies in quickly discovering counterexamples that are difficult to find by representing the entire winning or losing region as a BDD. Unbounded synthesis has similar advantages except that we can extend this advantage to specifications that are realisable.

\section{Summary}

I have now shown how to extend bounded realisability to unbounded games by interpolating abstract game trees to learn an overapproximation of the environment's winning region. The resulting algorithm is a sound and complete procedure for realisability that is efficient is certain cases where BDD based methods are not. In the next chapter I will present an implementation of the algorithm and show results.

\begin{itemize}
    \item Chapter~\ref{ch:bounded} introduced bounded realisability, which constructs game trees as abstractions of the game. An optimisation was described that prunes the search tree by learning the set of states that are losing for a particular abstraction. In this chapter states are learned from the same game trees with interpolation.
    \item Learning with interpolants ensures that certain properties are maintained on the losing states. By carefully maintaining invariants using those properties a fixed point in environment losing states can be detected.
    \item The constructed fixed point is an overapproximation of the total set of environment losing states. Unbounded synthesis can be more efficient than BDD solvers in cases where computing the entire set of states is costly.
\end{itemize}
