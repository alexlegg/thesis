\chapter{Unbounded Realisability}
\label{ch:unbounded}

\newtheorem*{exmpInt}{Example: Why we use interpolants}

In previous chapters I outlined an algorithm to solve bounded realisability games and an extension that can extract strategies from the result. Bounded realisability can be used to prove the existence of a winning strategy for the environment on the unbounded game by providing a witness. For the controller, the strongest claim that can be made is that the strategy is winning as long as the game does not extend beyond the maximum bound. The work described in this chapter can be used to address this by presenting another extension to the algorithm that solves unbounded realisability games.

The baseline solution to this problem is to set a maximum bound such that all runs in the unbounded game will be considered. The na\"ive approach is to use size of the state space as the bound ($2^\mathcal{S}$) so that all states may be explored by the algorithm. A more nuanced approach is to use the diameter of the game \cite{Biere99}, which is the smallest number $d$ such that for any state $x$ there is a path of length $\leq d$ to all other reachable states. Computing the diameter, however, is also expensive and solving a game bounded to the size of the diameter may be infeasible.

Instead I present an approach that iteratively solves games of increasing
bound while learning bad states from abstract games using Craig interpolation. We utilise the approximation properties of the interpolant to construct sets of states that underapproximate the total losing set for the controller. By underapproximating we avoid constructing a potentially large representation of this set that could be the cause of infeasibility in a BDD solver. Later in this chapter we will see that a careful construction of approximate sets enables a fixed point that is sufficient to prove the nonexistence of an environment-winning strategy.

\section{Algorithm}

Recall the bounded realisability algorithm from Chapter~\ref{ch:bounded}. A safety game is solved with respect to a bound on the number of game rounds. The algorithm is a counterexample guided approach that constructs abstractions of the game in the form of game trees. An abstract game tree restricts one player to the actions that label edges in the tree. In Chapter~\ref{ch:strategy} I presented a strategy extraction procedure that utilises the abstract game tree as a certificate for the game. In this chapter I introduce a similar approach that extracts an approximation of winning states from the certificate tree.

The bounded algorithm is reproduced in Algorithm~\ref{alg:unboundedBounded} with some modifications. The algorithm solves a game $(\mathcal{S}, \mathcal{U}, \mathcal{C}, \delta, s_0)$ with state variables $\mathcal{S}$, environment variables $\mathcal{U}$, controller variables $\mathcal{C}$, transition relation $\delta$, and initial states $s_0$. The safety condition of the game is defined by a set of error states $E$ that the controller must avoid and the environment must reach. The unbounded algorithm is modified to learn states when a candidate cannot be found for an abstract game (line~\ref{alg:unboundedBounded:learn}).

\begin{algorithm}
    \begin{algorithmic}[1]
        \Function{solveAbstract}{$p, s, k, T$}
        \State $cand \gets $ \Call{findCandidate}{$p, s, k, T$} 
        \IIf{$k = 1$} \Return $cand$ \EndIIf 
        \State $T' \gets T$
        \Loop
            \IIf{$cand = \texttt{NULL}$} \Return $\texttt{NULL}$ \EndIIf 
            \State $\langle cex, l, u \rangle \gets $ \Call{verify}{$p, s, k, T, cand$} 
            \IIf{$cex = \False$} \Return $cand$ \EndIIf 
            \State $T' \gets $ \Call{append}{$T', l, u$} 
            \State $cand \gets $ \Call{solveAbstract}{$p, s, k, T'$} 
        \EndLoop
        \EndFunction
        \algstore{b1}
    \end{algorithmic}

    \begin{algorithmic}
        \algrestore{b1}
        \Function{findCandidate}{$p, s, k, T$}
        \State $\hat{T} \gets $ \Call{extend}{$T$}
            \State $f \gets $ \IfElse{$p = \texttt{cont}$}{\Call{treeFormula}{$k, \hat{T}$}}{\Call{\textoverline{treeFormula}}{$k, \hat{T}$}} \EndIfElse
            \State $sol \gets $ \Call{SAT}{$s(X_{\hat{T}}) \land f$}
            \If{$sol = \texttt{unsat}$} 
                \LineComment{Unbounded solver learns states here}
                \State \IfElse{$p = \texttt{cont}$}{\Call{learn}{$s, \hat{T}$}}{\Call{\textoverline{learn}}{$s, \hat{T}$}}\label{alg:unboundedBounded:learn} \EndIfElse
                \State \Return $\texttt{NULL}$
            \Else
                \State \Return $\{ \langle n, c \rangle | n \in $ \Call{nodes}{$T$} $, c = \Call{sol}{n} \}$
            \EndIf
        \EndFunction
        \algstore{b2}
    \end{algorithmic}

    \begin{algorithmic}
        \algrestore{b2}
        \Function{verify}{$p, s, k, T, cand$}
            \For{$l \in leaves(gt)$}
            \State $\langle k', s'\rangle \gets $ \Call{outcome}{$s, k, cand, l$}
            \State $T' \gets$ \IfElse{$p = \textsc{cont}$}{$\emptyset$}{$\{ cand(l) \}$ } \EndIfElse
                \State $a \gets $ \Call{solveAbstract}{\Call{opponent}{$p$}, $s'$, $k'$, $T'$}
                \IIf{$a \neq \texttt{NULL}$} \Return $\langle \True, l, a \rangle$ \EndIIf
            \EndFor
            \State \Return $\langle \False, \emptyset, \emptyset \rangle$
        \EndFunction
    \end{algorithmic}

    \caption{Unbounded realisability}
    \label{alg:unboundedBounded}
\end{algorithm}

%%%The states in an abstract game with no controller candidate are
%%%\emph{must-losing}. The environment can always force the game into the error
%%%states from these states.

%%%From abstractions with no environment candidate we record the complement of
%%%states in the tree as \emph{may-losing}. The environment cannot reach the error
%%%state in a number of steps equal to the distance to the bottom of the tree.

%%%We maintain a set of states for each rank up to the current bound. We maintain
%%%an invariant over these sets via careful construction so that they are
%%%monotonically increasing by rank. We also ensure that the environment is unable
%%%to force play from one set to the next. Due to these invariants, when two
%%%adjacent sets become equivalent we know that the algorithm has reached a fixed
%%%point and the controller is winning in the unbounded game (line 6).

We extend the bounded synthesis algorithm to learn states losing for one of the players from failed attempts to find candidate strategies.  The learning procedure kicks in whenever \textsc{findCandidate} cannot find a candidate strategy for an abstract game tree. We can learn additional losing states from the tree via interpolation.  This is achieved in line~\ref{alg:unboundedBounded:learn} in Algorithm~\ref{alg:unboundedBounded}, enabled in the unbounded version of the algorithm, which invokes \textsc{learn} or \textsc{\textoverline{learn}} to learn controller or environment losing states respectively (Algorithm~\ref{alg:learn}).  

%%%For the controller, this means the environment partial strategy represented by $T$
%%%will always reach $E$ from $s$ no matter the assignments to $C$ variables (the
%%%bound is irrelevant).  For the environment, the controller can avoid $E$ for $k$
%%%rounds no matter the assignments to $U$ variables. Hence, we have established
%%%that $s$ is a losing state either always (for the controller) or for bound $k$
%%%(for the environment). 



\subsection{Example}

Consider a simple arbiter system in which the environment makes a request for a number of resources (1 or 2), and the controller may grant access to up to two resources.  The total number of requests grows each round by the number of environment requests and shrinks by the number of resources granted by the controller in the previous round.  The controller must ensure that the number of unhandled requests does not accumulate to more than 2.  Figure~\ref{fig:example} shows the variables (\ref{fig:examplevars}), the initial state of the system (\ref{fig:exampleinit}), and the formulas for computing next-state variable assignments (\ref{fig:exampletrans}) for this example. We use primed identifiers to denote next-state variables and curly braces to define the domain of a variable.


\begin{figure}
    \begin{subfigure}[t]{\textwidth}
        \centering
        \begin{tabular}{l | l | l}
            \textbf{Uncontrollable} & \textbf{Controllable} & \textbf{State} \\
            \hline
            \texttt{request = \{1,2\}} & \texttt{grant0 = \{0,1\}} & \texttt{resource0 = \{0,1\}} \\
            & \texttt{grant1 = \{0,1\}} & \texttt{resource1 = \{0,1\}} \\
            & & \texttt{nrequests = \{0,1,2,3\}} \\
        \end{tabular}
        \caption{Variables}
        \label{fig:examplevars}
    \end{subfigure}

    \vspace{5mm}
    \begin{subfigure}[t]{\textwidth}
        \centering
        \texttt{resource0 = 0; resource1 = 0; nrequests = 0;}
        \caption{Initial State}
        \label{fig:exampleinit}
    \end{subfigure}

    \begin{subfigure}[t]{\textwidth}
        \begin {align*}
            \texttt{resource0'} & \texttt{ = grant0;} \\
            \texttt{resource1'} & \texttt{ = grant1;} \\
            \texttt{nrequests'} & \texttt{ = (nrequests + request >= resource0 + resource1)} \\ 
                                & \texttt{ ? (nrequests + request - resource0 - resource1)}\\
                                & \texttt{ : 0;}
        \end{align*}
        \caption{Transition Relation}
        \label{fig:exampletrans}
    \end{subfigure}
    \caption{Example}
    \label{fig:example}
\end{figure}


\begin{figure}[t]
    \centering
    \begin{subfigure}[t]{.32\textwidth}
        \centering

        \begin{tikzpicture}[sibling distance = 18mm, level distance = 14mm]
            \node [circle,draw] (root){}
                child {node [circle,draw] {}
                    child {node [circle,draw,right=5pt] {}
                        edge from parent node [left=4pt,text width=1cm] {\texttt{gr0=1 gr1=0}}
                    }
                    child {node [circle,draw,left=5pt] {}
                        edge from parent node [right=2pt,text width=1cm] {\texttt{gr0=1 gr1=1}}
                    }
                    node [left=5pt] {$n$}
                    edge from parent node [above left=-6pt and 2pt,text width=1cm] {\texttt{gr0=0 gr1=1}}
                }
                child {node [circle,draw] (n) {}
                    child {node [circle,draw] {}
                        edge from parent node [right=2pt,text width=1cm] {\texttt{gr0=1 gr1=1}}
                    }
                    edge from parent node [above right=-6pt and 2pt,text width=1cm] {\texttt{gr0=1 gr1=1}}
                }
                node [above=4pt] {$\langle s, k \rangle$};
        \end{tikzpicture}
        \caption{A losing AGT $T$}
        \label{fig:interpolatetree}
    \end{subfigure}

    \begin{subfigure}[t]{.5\textwidth}
        \centering
        \begin{tikzpicture}[sibling distance = 18mm, level distance = 14mm]
            \node [circle,draw] (root){}
                child {node [circle,draw] {}
                    node [left=5pt] {$n$}
                    edge from parent node [above left=-6pt and 2pt,text width=1cm] {\texttt{gr0=1 gr1=0}}
                }
                child {node [circle,draw] (n) {}
                    child {node [circle,draw] {}
                        edge from parent node [right=2pt,text width=1cm] {\texttt{gr0=1 gr1=1}}
                    }
                    edge from parent node [above right=-6pt and 2pt,text width=1cm] {\texttt{gr0=1 gr1=1}}
                }
                node [above=4pt] {$\langle s, k \rangle$};
        \end{tikzpicture}
        \caption{Tree slice $T_1$}
        \label{fig:treef1}
    \end{subfigure}%
    \begin{subfigure}[t]{.5\textwidth}
        \centering
        \begin{tikzpicture}[sibling distance = 18mm, level distance = 14mm]
            \node [circle,draw] {}
                child {node [circle,draw,right=5pt] {}
                    edge from parent node [left=6pt,text width=1cm] {\texttt{gr0=1 gr1=0}}
                }
                child {node [circle,draw,left=5pt] {}
                    edge from parent node [right=6pt,text width=1cm] {\texttt{gr0=1 gr1=1}}
                }
                node [left=5pt] {$n$};
        \end{tikzpicture}
        \caption{Tree slice $T_2$}
        \label{fig:treef2}
    \end{subfigure}
    \caption{Splitting of an abstract game tree by the learning procedure.}
    \label{fig:interpolanttrees}
\end{figure}

\begin{algorithm}[t]
    \begin{algorithmic}[1]
        \Require $s(X_T) \land $ \Call{treeFormula}{$k, T$} $\equiv \bot$
        \Require \emph{Must-invariant} holds
        \Ensure \emph{Must-invariant} holds
        \Ensure $s(X_T) \land B^M \not\equiv \bot$
        \Comment $s$ will be added to $B^M$
        \Function{learn}{$s, T$}
            \IIf{\Call{succ}{$T$} $= \emptyset$}
                \Return
            \EndIIf
            \State $n \gets $ non-leaf node with min height 
            \State $\langle T_1, T_2 \rangle \gets $ \Call{gtSplit}{$T, n$}
            \State $\mathcal{I} \gets $ \Call{interpolate}{$s(X_T) \land $ \Call{treeFormula}{$k, T_1$}, \Call{treeFormula}{$k, T_2$}}
            \State $B^M \gets B^M \lor \mathcal{I}$
            \State \Call{learn}{$s, T_1$}
        \EndFunction
        \algstore{learn}
    \end{algorithmic}
    
    \begin{algorithmic}[1]
        \algrestore{learn}
        \Require $s(X_T) \land $ \Call{\textoverline{treeFormula}}{$k, T$} $\equiv \bot$
        \Require \emph{May-invariant} holds
        \Ensure \emph{May-invariant} holds
        \Ensure $s(X_T) \land B^m[$\Call{height}{$k, T$}$] \equiv \bot$
        \Comment{$s$ will be removed from $B^m$}
        \Function{\textoverline{learn}}{$s, T$}
            \IIf{\Call{succ}{$T$} $= \emptyset$}
                \Return
            \EndIIf
            \State $n \gets $ non-leaf node with min height 
            \State $\langle T_1, T_2 \rangle \gets $ \Call{gtSplit}{$T, n$}
            \State $\mathcal{I} \gets $ \Call{interpolate}{$s(X_T) \land $ \Call{\textoverline{treeFormula}}{$k, T_1$}, \Call{\textoverline{treeFormula}}{$k, T_2$}}
            \For{$i = 1$ to \Call{height}{$k, n$}}
                \State $B^m[i] \gets B^m[i] \setminus \mathcal{I}$
            \EndFor
            \State \Call{\textoverline{learn}}{$s, T_1$}
        \EndFunction
    \end{algorithmic}
    \caption{Learning algorithms}
    \label{alg:learn}
\end{algorithm}


This example is the $n=2$ instance of the more general problem of an arbiter of $n$ resources. For large values of $n$, the set of winning states has no compact representation, which makes the problem hard for BDD solvers. One approach is to compute the uncontrollable predecessor of the set of error states, i.e. the states from which the environment can force the game into an error state. In the example, one iteration of this operation would give the set: $\texttt{nrequests} = 3 \lor (\texttt{nrequests} = 2 \land ( \texttt{resource0} = 0 \lor \texttt{resource1} = 0)) \lor (\texttt{nrequests} = 1 \land (\texttt{resource0} = 0 \land \texttt{resource1} = 0))$. We try to avoid computing the entire winning set by employing interpolation to approximate it instead. The benefit of interpolation is that an approximation can be obtained efficiently from the resolution proof of two mutually unsatisfiability formulas.

Consider node $n$ in Figure~\ref{fig:interpolatetree}, which shows an abstract game tree for which the environment has no winning action. At this node there are two controller actions that prevent the environment from forcing the game into an error state in one game round. We want to use this tree to learn the states from which the controller can win playing one of these actions.

Given two formulas $F_1$ and $F_2$ such that $F_1 \land F_2$ is unsatisfiable, it is possible to construct a Craig interpolant $\mathcal{I}$ such that $F_1 \to \mathcal{I}$, $F_2 \land \mathcal{I}$ is unsatisfiable, and $\mathcal{I}$ refers only to the intersection of variables in $F_1$ and $F_2$. 

We choose a non-leaf node $n$ of $T$ with maximal depth, i.e., a node whose children are leafs (Algorithm~\ref{alg:learn}, line~3). We then split the tree at $n$ such that both slices $T_1$ and $T_2$ contain a copy of $n$ (line~4).  Figure~\ref{fig:treef1} shows $T_1$, which contains all of $T$ except $n$'s children, and $T_2$ (Figure~\ref{fig:treef2}), which contains only $n$ and its children.  There is no candidate strategy for $T$ so $s \land \textsc{\textoverline{treeFormula}}(k, T)$ is unsatisfiable.  By construction, $ (\textsc{\textoverline{treeFormula}}(k, T_1) \land \textsc{\textoverline{treeFormula}}(k, T_2)) \implies \textsc{\textoverline{treeFormula}}(k, T)$ and so we know that $(s \land$ $\textsc{\textoverline{treeFormula}}(k, T_1) \land \textsc{\textoverline{treeFormula}}(k, T_2))$ is also unsatisfiable.  

%%%This enables the construction of an interpolant that captures losing states at
%%%node $n$.

We construct an interpolant with $F_1 = s(X_T) \land \textsc{treeFormula}(k, T_1)$ and $F_2 = \textsc{treeFormula}(k, T_2)$ (line~5). The only variables shared between $F_1$ and $F_2$ are the state variable copies belonging to node $n$. By the properties of the interpolant, $F_2 \land \mathcal{I}$ is unsatisfiable, therefore all states in $\mathcal{I}$ are losing against abstract game tree $T_2$ in Figure~\ref{fig:treef2}.  We also know that $F_1 \to \mathcal{I}$, thus $\mathcal{I}$ contains all states reachable at $n$ by following $T_1$ and avoiding error states.  


At node $n$, the interpolant $(\texttt{nrequests} = 1 \land
\texttt{resource1} = 1)$ captures the information we need. Any action by the
environment followed by one of the controller actions at $n$ will be
winning for the controller.

We have discovered a set $\mathcal{I}$ of states losing for the environment.
Environment-losing states are only losing for a particular bound: given that
there does not exist an environment strategy that forces the game into an error
state in $k$ rounds or less; there may still exist a longer environment-winning
strategy.  We therefore record learned environment-losing states along with
associated bounds.  To this end, we maintain a conceptually infinite array of
sets $B^m[k]$ that are may-losing for the controller, indexed by bound $k$.
$B^m[k]$ are initialised to $E$ for all $k$.  Whenever an environment-losing
set $\mathcal{I}$ is discovered for a node $n$ with bound $\textsc{height}(k,
n)$ in line~13 of Algorithm~\ref{alg:learn}, this set is subtracted from
$B^m[i]$, for all $i$ less than or equal to the bound (lines~14--16).

The \textsc{\textoverline{treeFormula}} function is modified for the unbounded
solver (Algorithm~\ref{alg:unboundedTreeFormula}) to constrain the environment
to the appropriate $B^m$. This enables further interpolants to be constructed
by the learning procedure recursively splitting more nodes from $T_1$
(Algorithm~\ref{alg:learn}, line~7) since the states that are losing to $T_2$
are no longer contained in $B^m$.

\begin{algorithm}[t] \caption{Amended tree formulas for Controller and
    Environment} \label{alg:unboundedTreeFormula} \begin{algorithmic}[1]
        \Function{treeFormula}{$k, T$} \If{$\Call{height}{k, T} = 0$} \State
        \Return{ $\lnot \Call{$B^M$}{X_{T}}$ } \Else \State \Return{$\lnot
            \Call{$B^M$}{X_{T}} \land$ \\ $$\bigwedge_{\langle e, n \rangle \in
            \Call{succ}{T}}(\Call{$\delta$}{X_T, U_e, C_n, X_n} \land U_e =
            \Call{action}{e} \land \Call{treeFormula}{k, n})$$ } \EndIf
        \EndFunction \algstore{tf1} \end{algorithmic}

    \begin{algorithmic}[1]
        \algrestore{tf1}
        \Function{\textoverline{treeFormula}}{$k, T$}
        \If{$\Call{height}{k, T} = 0$}
        \State \Return{\Call{E}{$X_T$}}
        \Else
        \State \Return{ \Call{$B^m[\Call{height}{k, T}]$}{$X_{T}$} $\land$ \\
            $$\bigg( \Call{E}{X_T} \lor \bigvee_{\langle e, n \rangle \in \Call{succ}{T}}(\Call{$\delta$}{X_T, U_n, C_e, X_n} \land C_e = \Call{action}{e} \land \Call{\textoverline{treeFormula}}{k, n})\bigg)$$ }
        \EndIf
        \EndFunction
    \end{algorithmic}
\end{algorithm}

Learning of states losing for the controller is similar (\textsc{learn} in Algorithm~\ref{alg:learn}). The main difference is that environment-losing states are losing for all bounds. Therefore we record these states in a single set $B^M$ of must-losing states (Algorithm~\ref{alg:learn}, line~6).  This set is initialised to the error set $E$ and grows as new losing states are discovered.  The modified \textsc{\textoverline{treeFormula}} function (Algorithm~\ref{alg:unboundedTreeFormula}) blocks must-losing states, which also allows for recursive learning over the entire tree.

\subsection{Main synthesis loop}

Figure~\ref{alg:unbounded} shows the main loop of the unbounded synthesis algorithm.
The algorithm invokes the modified bounded synthesis procedure with increasing bound $k$
until the initial state is in $B^M$ (environment wins) or $B^m$ reaches a fixed point 
(controller wins). We prove correctness in the next section.

\begin{algorithm}[h]
    \begin{algorithmic}[1]
        \Function{solveUnbounded}{$T$}
            \State $B^M \gets E$
            \State $B^m[0] \gets E$
            \For{$k = 1 \dots$}
                \If{\Call{SAT}{$I \land B^M$}}
                    \LineComment{Losing in the initial state}
                    \Return \texttt{unrealisable} 
                \EndIf
                \IIf{$\exists i < k . \  B^m[i] \equiv B^m[i+1]$} \Comment{Reached fixed point}
                    \State \hspace{\algorithmicindent} \Return \texttt{realisable} 
                \EndIIf
                \State $B^m[k] \gets E$
                \State \Call{checkBound}{$k$}
            \EndFor
        \EndFunction
        \algstore{u1}
    \end{algorithmic}

    \begin{algorithmic}
        \algrestore{u1}
        \Require \emph{May} and \emph{must} invariants hold
        \Ensure \emph{May} and \emph{must} invariants hold
        \Ensure $I \not\in B^m[k]$ if there exists a winning controller strategy with bound $k$
        \Ensure $I \in B^M$ if there exists a winning environment strategy with bound $k$
%%%        $\exists u_{k..1} \forall c_{k..1} \  s(x_k) \land (E_k \lor (\delta(x_k, u_k, c_k) \land E(x_{k-1}) \lor ... \land E(x_1) \lor (\delta(x_1, u_1, c_1) \land E(x_0))...)$
%%%            $\forall u_{k..1} \exists c_{k..1} \  s(x_k) \land \lnot E_k \land \delta(x_k, u_k, c_k) \land \lnot E(x_{k-1}) \land ... \land \delta(x_1, u_1, c_1) \land \lnot E(x_0)$
        \Function{checkBound}{$k$}
            \State \Return \Call{solveAbstract}{$\texttt{env}, I, k, \emptyset$}
        \EndFunction
    \end{algorithmic}
    \caption{Unbounded Synthesis}
    \label{alg:unbounded}
\end{algorithm}



\section{Correctness}


We define two global invariants of the algorithm.  The \emph{may-invariant}
states that sets $B^m[i]$ grow monotonically with $i$ and that each $B^m[i+1]$
overapproximates the states from which the environment can force the game into
$B^m[i]$. We call this operation $Upre$, the uncontrollable predecessor. So the
\emph{may-invariant} is: $$\forall i<k.~B^m[i] \subseteq B^m[i+1], Upre(B^m[i])
\subseteq B^m[i+1].$$

The \emph{must-invariant} guarantees that the must-losing set $B^M$ is an
underapproximation of the actual losing set $B$: $$B^M \subseteq B.$$

Correctness of $\textsc{solveUnbounded}$ follows from these invariants. The
must-invariant guarantees that the environment can force the game into an error
state from $B^M$, therefore checking whether the initial state is in $B^M$ (as
in line~5) is sufficient to return \texttt{unrealisable}. The may-invariant
tells us that if $B^m[i] \equiv B^m[i+1]$ (line~6) then $Upre(B^m[i]) \subseteq
B^m[i]$, i.e. $B^m[i]$ overapproximates the winning states for the environment.
We know that $I \not\in B^m[k]$ due to the post-condition of
\textsc{checkBound}, and since the may-invariant tells us that $B^m$ is
monotonic then $I$ must not be in $B^m[i]$. If $I \not\in B^m[i]$ then $I$ is
not in the winning states for the environment and the controller can always win
from $I$. 

Both invariants trivially hold after $B^m$ and $B^M$ have been initialised in
the beginning of the algorithm. The sets $B^m$ and $B^M$ are only modified by
the functions \textsc{learn} and \textoverline{\textsc{learn}}.  Below we prove
that \textoverline{\textsc{learn}} maintains the invariants.  The proof of
\textsc{learn} is similar.

\subsection{Proof of \textoverline{\textsc{learn}}}

We prove that postconditions of \textsc{\textoverline{learn}} are satisfied
assuming that its preconditions hold.

Line~(11--12) splits the tree $T$ into $T_1$ and $T_2$, such that $T_2$ has depth
1.  Consider formulas $F_1=s(X_T) \land
\textsc{\textoverline{treeFormula}}(k, T_1)$ and $F_2 =
\textsc{\textoverline{treeFormula}}(k, T_2)$.  These formulas only share variables
$X_n$.  Their conjunction $F_1 \land F_2$ is unsatisfiable, as by construction
any solution of $F_1 \land F_2$ also satisfies $s(X_T) \land
\textsc{\textoverline{treeFormula}}(k, T)$, which is unsatisfiable (precondition (b)).  Hence the
interpolation operation is defined for $F_1$ and $F_2$.  

Intuitively, the interpolant computed in line~(13) overapproximates the set of
states reachable from $s$ by following the tree from the root node to $n$,
and underapproximates the set of states from which the environment loses
against tree $T_2$.  

Formally, $\II$ has the property $\II \land F_2 \equiv \bot$.  Since $T_2$ is
of depth 1, this means that the environment cannot force the game into
$B^m[\textsc{height}(k, n)-1]$ playing against the counterexample moves in $T_2$.
Hence, $\II \cap Upre(B^m[\textsc{height}(k, n)-1]) = \emptyset$.  Furthermore,
since the may-invariant holds, $\II \cap Upre(B^m[i]) =
\emptyset$, for all $i < \textsc{height}(k, n)$.  Hence, removing $\II$ from all
$B^m[i], i\leq \textsc{height}(k, n)$ in line~(15) preserves the may-invariant,
thus satisfying the first post-condition.

Furthermore, the interpolant satisfies $F_1 \rightarrow \II$, i.e., any
assignment to $X_n$ that satisfies $s(X_T) \land
\textsc{\textoverline{treeFormula}}(k, T_1)$ also satisfies $\II$.  Hence,
removing $\II$ from $B^m[\textsc{height}(k, n)]$ makes $s(X_T) \land
\textsc{\textoverline{treeFormula}}(k, T_1)$ unsatisfiable, and hence all
preconditions of the recursive invocation of \textsc{\textoverline{learn}} in
line~(17) are satisfied.  

At the second last recursive call to \textsc{\textoverline{learn}}, tree $T_1$
is empty, $n$ is the root node, $\textsc{\textoverline{treeFormula}}(k, T_1)
\equiv B^m[\textsc{height}(k, T_1)](X^T)$; hence $s(X_T) \land
\textsc{\textoverline{treeFormula}}(k, T_1) \equiv s(X_T) \land
B^m[\textsc{height}(k, T_1)](X^T) \equiv \bot$.  Thus the second postcondition of
\textsc{\textoverline{learn}} holds.

The proof of \textsc{learn} is similar to the above proof of \textsc{learn}. An
interpolant constructed from $F_1=s(X_T) \land \textsc{treeFormula}(k, T_1)$
and $F_2 = \textsc{treeFormula}(k, T_2)$ has the property $\II \land F_2 \equiv
\bot$ and the precondition ensures that the controller is unable to force the
game into $B^M$ playing against the counterexample moves in $T_2$. Thus adding
$\II$ to $B^M$ maintains the must-invariant satisfying the first postcondition.

Likewise, in the second last recursive call of \textsc{learn} with the empty
tree $T_1$ and root node $n$: $\textsc{treeFormula}(k, T_1) \equiv \lnot
B^M(X_T)$.  Hence $s(X_T) \land \textsc{treeFormula}(k, T_1) \equiv
s(X_T) \land \lnot B^M(X_T) \equiv \bot$. Therefore $s(X_T) \land
B^M(X_T) \not\equiv \bot$, the second postcondition, is true.

\subsection{Proof of Termination}

We must prove that $\textsc{checkBound}$ terminates and that upon termination
its postcondition holds, i.e., state $I$ is removed from $B^m[\kappa]$ if there
is a winning controller strategy on the bounded safety game of maximum bound
$\kappa$ or it is added to $B^M$ otherwise. Termination follows from
completeness of counterexample guided search, which terminates after
enumerating all possible opponent moves in the worst case.

Assume that there is a winning strategy for the controller at bound $\kappa$.
This means that at some point the algorithm discovers a counterexample tree of
bound $\kappa$ for which the environment cannot force into $E$. The algorithm
then invokes the \textsc{\textoverline{learn}} method, which removes $I$ from
$B^m[\kappa]$.  Alternatively, if there is a winning strategy for the
environment at bound $\kappa$ then a counterexample losing for the controller
will be found.  Subsequently \textsc{learn} will be called and $I$ added to
$B^M$.

\section{Optimisation: Generalising the initial state}

This optimisation allows us to learn may and must losing states faster.
Starting with a larger set of initial states we increase the reachable set and
hence increase the number of states learned by interpolation. This optimisation
requires a modification to $\textsc{solveAbstract}$ to handle sets of states,
which is not shown.

The optimisation is relatively simple and is inspired by a common greedy
heuristic for minimising $\texttt{unsat}$ cores. Initial state $I$ assigns a value to
each variable in $X$. If the environment loses $\langle I, k
\rangle$ then we attempt to solve for a generalised version of $I$ by removing
one variable assignment at a time. If the environment loses from the larger set of
states then we continue generalising. In this way we learn more
states by increasing the reachable set. In our benchmarks we have observed that
this optimisation is beneficial on the first few iterations of
\textsc{checkBound}.

\begin{algorithm}
    \begin{algorithmic}
        \Function{checkBound}{$k$}
            \State $r \gets $ \Call{solveAbstract}{$\texttt{env}, I, k, \emptyset$}
            \IIf{$r \neq \emptyset$} \Return $r$ \EndIIf
            \State $s' \gets I$
            \For{$x \in X$}
            \State $r \gets$ \Call{solveAbstract}{$\texttt{env}, s' \setminus \{x\}, k, \emptyset$} 
                \If{$r = \texttt{NULL}$}
                    \LineComment{Remove the assignment to $x$ from $s'$}
                    \State $s' \gets s' \setminus \{x\}$ 
                \EndIf 
            \EndFor
            \State \Return $\texttt{NULL}$
        \EndFunction
    \end{algorithmic}
    \caption{Generalise $I$ optimisation}
    \label{alg:opt1}
\end{algorithm}


\section{Discussion}

Unbounded realisability is designed to take advantage of the strengths of bounded realisability but provide the completeness offered by a fixed point computation. These conflicting aims are addressed by overapproximation via interpolation, which allows completeness without sacrificing performance to an exploding symbolic representation.

\subsection{Related work}

Synthesis of safety games is a thoroughly explored area of research with most
efforts directed toward solving games with BDDs \cite{Burch90} and abstract
interpretation \cite{Walker14,Brenguier14}. Satisfiability solving has been used
previously for synthesis in a suite of methods proposed by Bloem et al.
\cite{Bloem14}. The authors propose employing competing SAT solvers to learn
clauses representing bad states, which is similar to our approach but does not
unroll the game.  They also suggest QBF solver, template-based, and Effectively
Propositional Logic (EPR) approaches.

SAT-based bounded model checking approaches that unroll the transition relation
have been extended to unbounded by using conflicts in the solver
\cite{McMillan02}, or by interpolation \cite{McMillan03}. However, there
are no corresponding adaptations to synthesis.

Incremental induction \cite{Bradley11} is another technique for unbounded
model checking that inspired several approaches to synthesis including the work
presented here.  Morgenstern et al.~\cite{Morgenstern13} proposed an
technique that computes sets of states that overapproximate the losing states
(similar to our $B^m$) and another set of winning states (similar to the
negation of $B^M$).  Their algorithm maintains a similar invariant over the
sets of losing states as our approach and has the same termination condition.
It differs in how the sets are computed, which it does by inductively proving
the number of game rounds required by the environment to win from a state.
Chiang and Jiang~\cite{Chiang15} recently proposed a similar approach that
focusses on computing the winning region for the controller forwards from the
initial state in order to take advantage of reachability information and bad
transition learning without needing to discard learnt clauses.

There are approaches to synthesis of LTL specifications that use bounds to simplify the problem.  The authors of \cite{Finkbeiner13} suggest a methodology directly inspired by bounded model checking and it has been adapted to symbolic synthesis \cite{Ehlers12}. In contrast to the approach here the bound is placed on the implementation instead of the number of game rounds.  Lazy synthesis \cite{Finkbeiner12} similarly constructs implementations of a bounded size but does so in a counterexample guided approach. Their approach is to use an SMT solver to produce a candidate implementation and then check the implementation with a BDD based model checker. These bounded synthesis techniques are similar in idealogy to the approach described here but are used to solve a different problem.

%%%The original bounded synthesis algorithm of Narodytska et
%%%al.~\cite{narodytska14} solves realisability without constructing a strategy.
%%%In \cite{een2015} the realisability algorithm is extended with strategy
%%%extraction. The technique relies on interpolation over abstract game trees to
%%%compute the winning strategy.  In the present work we use interpolation in a
%%%different way in order to learn losing states of the game. In addition, this
%%%method could be easily adapted to the unbounded realisability algorithm
%%%presented here to generate unbounded strategies.

\subsection{Limitations}

Bounded synthesis is generally efficient for games without a high branching factor, as discussed in Chapter~\ref{ch:bounded}. Learning states from the game tree in order to extend to unbounded synthesis does not introduce significant complexity to the problem. However, since the learned states are approximations it may take many iterations of increasing the bound to converge on a fixed point.

\subsection{Strengths}

\section{Summary}
